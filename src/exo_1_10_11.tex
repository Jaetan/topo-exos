\documentclass[11pt,a4paper,twoside]{article}
\usepackage{mathtools}
\usepackage{amsfonts}
\usepackage{amssymb}
\usepackage{amsthm}
\usepackage{mathrsfs}
\usepackage[shortlabels]{enumitem}
\usepackage{parskip}
\setlength{\parindent}{15pt}

\theoremstyle{definition}
\newcounter{excounter}
\setcounter{excounter}{10}
\newtheorem{exercise}[excounter]{Exercise}

\begin{document}

\begin{exercise}

  Let $A$ and $B$ be two sets. Using the well-ordering theorem, prove that either they have the same cardinality,
  or one has cardinality greater than the other. [\emph{Hint:} if there is no surjection $f : A \to B$, apply
  the preceding exercise.]

\end{exercise}

\begin{proof}

  If there is a bijection from $A$ to $B$, then they have the same cardinality. Suppose therefore that there is
  no such bijection.

  From the well-ordering theorem, there exist order relations such that $A$ and $B$ are well-ordered. Let us consider
  $A$ and $B$ as well-ordered sets.

  If there is a surjection from a section of $A$ onto $B$, then using the axiom of choice we can define an injection
  of $B$ into $A$. If there also exists an injection of $A$ into $B$, then there exists a bijection between $A$ and $B$,
  which is contrary to our hypothesis. Therefore there is no injection of $A$ into $B$, and $A$ has greater cardinality than $B$.

  Suppose there is no surjection from a section of $A$ onto $B$. Using the previous exercise, we can define a function
  $h : A \to B$ such that
  \begin{equation} \label{eq:induction}
    \forall x \in A, \quad h ( x ) = \min ( B - h ( S_x ) )
  \end{equation}
  Let $J$ be the subset of $A$ such that $h|J$ is injective, and suppose that there exists $x \in A - J$.
  Since $A$ is well-ordered, there is a smallest such element $\beta$.

  From \eqref{eq:induction} we deduce that $h ( \beta ) \notin h ( S_\beta )$, so that $\beta \in J$, which is a contradiction.
  Therefore $J = A$ and $h$ is injective.

  If there also exists an injection from $B$ into $A$, then there exists a bijection between $A$ and $B$, which is contrary to
  our hypothesis. Therefore there is no such injection, and $B$ has greater cardinality than $A$.

\end{proof}

\end{document}
