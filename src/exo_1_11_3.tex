\documentclass[11pt,a4paper,twoside]{article}
\usepackage{mathtools}
\usepackage{amsfonts}
\usepackage{amssymb}
\usepackage{amsthm}
\usepackage{mathrsfs}
\usepackage[shortlabels]{enumitem}
\usepackage{parskip}
\setlength{\parindent}{15pt}

\theoremstyle{definition}
\newcounter{excounter}
\setcounter{excounter}{2}
\newtheorem{exercise}[excounter]{Exercise}

\newcounter{examplecounter}
\setcounter{examplecounter}{0}
\theoremstyle{plain}
\newtheorem{example}[examplecounter]{Example}

\begin{document}

\begin{exercise}

  Let $A$ be a set with a strict partial order $\prec$; let $x \in A$. Suppose that we wish to find
  a maximal simply ordered subset $B$ of $A$ that contains $x$. One plausible way of attempting to
  define $B$ is to let $B$ equal the set of all those elements of $A$ that are \emph{comparable} with $x$;
  \begin{equation*}
    B ~ \{ y \mid y \in A \text{ and either} x \prec y \text{ or } y \prec x \}
  \end{equation*}
  But this will not always work. In which if Examples 1 and 2 will this procedure succeed and in which
  will it not?

\end{exercise}

~\\
\begin{example}
  If $\mathscr{A}$ is any collection of sets, the relation ''is a proper subset of'' is a strict partial order on $\mathscr{A}$.
  Suppose that $\mathscr{A}$ is the collection of all circular regions (interiors of circles) in the plane. One maximal
  simply ordered subcollection consists of all the circular regions with center at the origin. Another maximal simply ordered
  subcollection consists of all circular regions bounded by circles tangent from the right to the $y$-axis at the origin.
\end{example}

~\\
\begin{example}
  If $( x_0, y_0 )$ and $( x_1, y_1 )$ are two points of the plane $\mathbb{R}^2$, define
  \begin{equation*}
    ( x_0, y_0 ) \prec ( x_1, y_1 )
  \end{equation*}
  if $y_0 = y_1$ and $x_0 < x_1$. This is a partial ordering of $\mathbb{R}^2$ under which two points are comparable only if
  they lie on the same horizontal line. The maximal simply ordered sets are the horizontal lines in $\mathbb{R}^2$.
\end{example}

\begin{proof}\hfill

  \begin{itemize}

  \item Let $\mathscr{C}_1, \mathscr{C}_2, \mathscr{C}$ be circular regions in the plane, such that $\mathscr{C}_1 \subset \mathscr{C}$,
    $\mathscr{C}_2 \subset \mathscr{C}$, and $\mathscr{C}_1 \cap \mathscr{C}_2 = \varnothing$. Let $x = \mathscr{C}$, we have
    $\mathscr{C}_1 \in B$ and $\mathscr{C}_2 \in B$, but $\mathscr{C}_1$ and $\mathscr{C}_2$ are not comparable by strict inclusion.
    Then $B$ is not simply ordered, so the procedure fails in this example.

  \item Let $( x_0, y_0 )$ be a point in $\mathbb{R}^2$. The only points comparable to $( x_0, y_0 )$ are those on the line
    $y = y_0$, which is a maximal simply ordered set for $\prec$. Therefore $B = \{ ( x, y ) \in \mathbb{R}^2 \mid y = y_0 \}$ is
    a maximal simply ordered set containing $( x_0, y_0 )$ in this case.

  \end{itemize}

\end{proof}

\end{document}
