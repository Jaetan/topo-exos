\documentclass[11pt,a4paper,twoside]{article}
\usepackage{mathtools}
\usepackage{amsfonts}
\usepackage{amssymb}
\usepackage{amsthm}
\usepackage{mathrsfs}
\usepackage[shortlabels]{enumitem}
\usepackage{parskip}
\setlength{\parindent}{15pt}

\theoremstyle{definition}
\newcounter{excounter}
\setcounter{excounter}{4}
\newtheorem{exercise}[excounter]{Exercise}

\theoremstyle{plain}
\newtheorem*{lemma}{Lemma}

\begin{document}

\begin{exercise}

  Show that Zorn's lemma implies the following:

\end{exercise}

~\\
\begin{lemma}[Kuratowski]
  Let $\mathscr{A}$ be a collection of sets. Suppose that for every subcollection
  $\mathscr{B}$ of $\mathscr{A}$ that is simply ordered by proper inclusion, the union of the
  elements of $\mathscr{B}$ belongs to $\mathscr{A}$. Then $\mathscr{A}$ has an element that is
  properly contained in no other element of $\mathscr{A}$.
\end{lemma}

\begin{proof}

  For all $\mathscr{B}$ subcollection of $\mathscr{A}$ that is simply ordered by proper inclusion,
  the set $B = \cup_{b \in \mathscr{B}} \,b$ is an upper bound of $\mathscr{B}$ in $\mathscr{A}$.
  Zorn's lemma gives us the existence of a maximal element $M \in \mathscr{A}$. Since $M$ is maximal,
  for all $A \in \mathscr{A}$ such that $A \neq M$, we have $A \subset M$. Therefore $M$ is not
  a proper subset of any other element of $\mathscr{A}$.

\end{proof}

\end{document}
