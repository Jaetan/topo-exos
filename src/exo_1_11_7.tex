\documentclass[11pt,a4paper,twoside]{article}
\usepackage{mathtools}
\usepackage{amsfonts}
\usepackage{amssymb}
\usepackage{amsthm}
\usepackage{mathrsfs}
\usepackage[shortlabels]{enumitem}
\usepackage{parskip}
\setlength{\parindent}{15pt}

\theoremstyle{definition}
\newcounter{excounter}
\setcounter{excounter}{6}
\newtheorem{exercise}[excounter]{Exercise}

\begin{document}

\begin{exercise}

  Show that the Tukey lemma implies the Hausdorff maximum principle. [\emph{Hint:} if
  $\prec$ is a strict partial order on $A$, let $\mathscr{A}$ be the collection of all
  subsets of $A$ that are simply ordered by $\prec$. Show that $\mathscr{A}$ is of
  finite type.]

\end{exercise}

\begin{proof}

  Let $\mathscr{A}$ be the collection of all subsets of $A$ that are simply ordered by $\prec$,
  let $B \in \mathscr{A}$, and let $C$ be a finite subset of $B$.
  As a subset of a simply ordered set , $C$ is simply ordered, (elements of $C$ are comparable
  as elements of $B$) and therefore $C \in \mathscr{A}$.
  From this we deduce that $\mathscr{A}$ is of finite type, and, from the Tukey lemma,
  there exists $M \in \mathscr{A}$ such that no element of $A$ properly contains $M$.
  The element $M$ is a simply ordered subset of $A$ such that for all $B \in \mathscr{A}$,
  $M \subset B \Rightarrow M = B$, and therefore $M$ is maximal in the sense of Hausdorff's
  maximum principle.

\end{proof}

\end{document}
