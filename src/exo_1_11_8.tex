\documentclass[11pt,a4paper,twoside]{article}
\usepackage{mathtools}
\usepackage{amsfonts}
\usepackage{amssymb}
\usepackage{amsthm}
\usepackage{mathrsfs}
\usepackage[shortlabels]{enumitem}
\usepackage{parskip}
\setlength{\parindent}{15pt}

\theoremstyle{definition}
\newcounter{excounter}
\setcounter{excounter}{7}
\newtheorem{exercise}[excounter]{Exercise}

\begin{document}

\begin{exercise}

  A typical use of Zorn's lemma in algebra is the proof that every vector space
  has a basis. Recall that is $A$ is a subset of the vector space $V$, we say
  a vector belongs to the \emph{span} of $A$ if it equals a finite linear combination
  of elements of $A$. The set $A$ is \emph{independent} if the only finite linear
  combination of elements of $A$ that equals the zero vector is the trivial one having
  all coefficients zero. If $A$ is independent and if every vector in $V$ belongs to
  the span of $A$, then $A$ is a \emph{basis} for $V$.
  \begin{enumerate}[(a)]

  \item If $A$ is independent and $v \in V$ does not belong to the span of $A$,
    show that $A \cup \{ v \}$ is independent.

  \item Show the collection of all independents sets in $V$ has a maximal element.

  \item Show that $V$ has a basis.

  \end{enumerate}

\end{exercise}

\begin{proof} \hfill

  If the only vector in $V$ is the zero vector, then $V$ does not have a basis. Therefore we
  suppose here that $V$ has at least one nonzero vector.

  \begin{enumerate}[(a)]

  \item Suppose that $A$ is an independent set of vectors of $V$. Suppose that $v \in V$
    is not in the span of $A$. and suppose that for some linear combination $c$ of vectors of $A$
    and some scalar $\alpha$, we have $\alpha \cdot v + c = 0$. If $\alpha = 0$, then $c = 0$,
    and since $A$ is independent, every coefficient in $c$ is also $0$; from this we deduce that
    $A \cup \{ v \}$ is independent. Otherwise, the inverse $\alpha^{-1}$ exists and we have
    $v = - \alpha^{-1} \cdot c$, a linear combination of vectors of $A$, so that $v$ is in the span
    of $A$, contrary to the hypothesis. Therefore $A \cup \{ v \}$ is independent.

  \item Let $\mathscr{A}$ be the set of all independent sets of vectors of $V$. Proper inclusion
    is a strict partial order on $\mathscr{A}$, so from the maximum principle, there exists a subset
    $M$ of $\mathscr{A}$ that is simply ordered and maximal. Let $U = \cup_{m \in M} \,m$,
    and let $v$ be a linear combination of vectors $v_1, \dotsc, v_n$ in $U$ such that $v = 0$.
    For all $i$, there exists $V_i \in M$ such that $v_i \in V_i$. The finite set $\{ V_1, \dotsc, V_n \}$,
    as a subset of the linearly ordered set $M$, is linearly ordered; therefore it has
    a maximal element $V_m$. For all $i \in \{ 1, \dotsc, n \}$, we have $v_i \in V_m$. As an element
    of $\mathscr{A}$, $V_m$ is independent. From this we conclude that all coefficients in the
    linear combination $v$ are zero, and therefore $U$ is independent.

    For all $m \in M$, we have $m \subset U$. Suppose that there exists $W \in \mathscr{A}$
    such that $U \subset W$. Then for all $m \in M$, we also have $m \subset W$, so that $M \cup W$
    is simply ordered by proper inclusion and strictly contains $M$, contrary to the fact that
    $M$ is maximal for proper inclusion. Therefore $U$ is a maximal element of $\mathscr{A}$.

  \item Let $U$ be the element defined in the above point, and let $v \in V$. Suppose that $v$ is
    not in the span of $U$; then the set $U \cup \{ v \}$ is independent. For all $B \subset U$,
    we have $B \subset U \cup \{ v \}$, so $U \cup \{ v \}$ is simply ordered, and properly contains $U$,
    contrary to the fact that $U$ is a maximal element of $\mathscr{A}$.
    Therefore $v$ is in the span of $U$. Since $U$ is independent, we deduce that $U$ is a basis for $V$.

  \end{enumerate}

\end{proof}

\end{document}
