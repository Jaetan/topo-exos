\documentclass[11pt,a4paper,twoside]{article}
\usepackage{mathtools}
\usepackage{amsfonts}
\usepackage{amssymb}
\usepackage{amsthm}
\usepackage{mathrsfs}
\usepackage[shortlabels]{enumitem}

\theoremstyle{definition}
\newcounter{excounter}
\setcounter{excounter}{1}
\newtheorem{exercise}[excounter]{Exercise}

\begin{document}

\begin{exercise}
Determine which of the following statements are true for all sets $A, B, C, D$.
\begin{enumerate}[(a)]
\item $A \subset B$ and $A \subset C \iff A \subset (B \cup C)$
\item $A \subset B$ or $A \subset C \iff A \subset (B \cup C)$
\item $A \subset B$ and $A \subset C \iff A \subset (B \cap C)$
\item $A \subset B$ or $A \subset C \iff A \subset (B \cap C)$
\item $A - (A - B) = B$
\item $A - (B - A) = A - B$
\item $A \cap (B - C) = (A \cap B) - (A \cap C)$
\item $A \cup (B - C) = (A \cup B) - (A \cup C)$
\item $(A \cap B) \cup (A - B) = A$
\item $A \subset C$ and $B \subset D$ $\implies A \times B \subset C \times D$
\item The converse of (j)
\item The converse of (j), assuming that $A$ and $B$ are nonempty
\item $(A \times B) \cup (C \times D) = (A \cup C) \times (B \cup D)$
\item $(A \times B) \cap (C \times D) = (A \cap C) \times (B \cap D)$
\item $A \times (B - C) = (A \times B) - (A \times C)$
\item $(A - B) \times (C - D) = (A \times C - B \times C) - A \times D$
\item $(A \times B) - (C \times D) = (A - C) \times (B - D)$
\end{enumerate}
\end{exercise}

\begin{proof}\hfill
\begin{enumerate}[(a)]

\item False.

  $A \subset B$ and $B \subset B \cup C$, so $A \subset (B \cup C)$, and we have ($A \subset B$ and $A \subset C \implies A \subset (B \cup C)$).
  Take $A = \{ 1, 2, 3 \}, B = \{ 1, 2 \}, C = \{ 3 \}$. Then $A \subset (B \cup C)$, but neither $A \subset B$ nor $A \subset C$, so the converse is false.

\item False.

  $A \subset B$ and $B \subset B \cup C$, so $A \subset (B \cup C)$, and we have ($A \subset B$ or $A \subset C \implies A \subset (B \cup C)$).
  The same counter-example shows that the converse is false here too.

\item True.

  Let $x \in A$. From $A \subset B$ we deduce $x \in B$. From $A \subset C$ we deduce $x \in C$. From both previous statements, we deduce $x \in B \cap C$.
  This being true for all $x \in A$, we deduce ($A \subset B$ and $A \subset C \implies A \subset (B \cap C)$).
  Conversely, from $A \subset (B \cap C)$ and $B \cap C \subset B$, we deduce $A \subset B$.
  From $A \subset (B \cap C)$ and $B \cap C \subset C$, we deduce $A \subset C$.
  From both, we deduce $(A \subset B \cap C) \implies A \subset B$ and $A \subset C$.

\item False.

  Take $A = \{ 1 \}, B = \{ 1 \}, C = \{ 2 \}$. We have ($A \subset B$ or $A \subset C$), but $A \not\subset B \cap C$ since $B \cap C = \varnothing$ and $A \not= \varnothing$.
  The converse is true, though. From $A \subset B \cap C$ and $B \cap C \subset B$ we deduce $A \subset B$. From it, we deduce $A \subset B$ or $A \subset C$.

\item False.

  Take $A = \{ 1, 2 \}, B = \{ 2, 3 \}$. Then $A - B = \{ 1 \}$ and $A - (A - B) = \{ 2 \} \not= B$.
  However $A - (A - B) \subset B$. Let $x \in A - (A - B)$. This is equivalent to $x \in A$ and $x \notin A - B$.
  Also, $x \notin A - B$ is equivalent to $x \notin A$ or $x \in B$.
  From both of these, we deduce that $A - (A - B) = A \cap B \subset B$.

\item False.

  Let $x \in B - A$. By definition of $B - A$, $x \in B$ and $x \notin A$. From this we deduce that $A \cap (B - A) = \varnothing$.
  Since the sets $A$ and $B - A$ are disjoint, $A - (B - A) = A$. Noting that $A - B \subset A$, we deduce that $A - B \subset A - (B - A)$.

\item True.

  Let $x \in A \cap (B - C)$. This is equivalent to $x \in A$ and $x \in B$ and $x \notin C$, so $A \cap (B - C) = (A \cap B) - C$.
  Let $x \in (A \cap B) - (A \cap C)$. This is equivalent to $x \in A$ and $x \in B$ and ($x \notin A$ or $x \notin C$), which simplifies to
  $x \in A$ and $x \in B$ and $x \notin C$, which is again $(A \cap B) - C$.

\item False.

  Let $x \in (A \cup B) - (A \cup C)$. This is equivalent to ($x \in A$ or $x \in B$) and ($x \notin A$ and $x \notin C$), which simplifies to
  $x \in B$ and $x \notin A$ and $x \notin C$, so that $(A \cup B) - (A \cup C) = (B - C) - A \subset A \cup (B - C)$.
  Taking $A = \{ 1 \}, B = \{ 2, 3 \}, C = \{ 3 \}$, we have $A \cup (B - C) = \{ 1, 2 \}$
  and $(A \cup B) - (A \cup C) = \{ 1, 2, 3 \} - \{ 1, 3 \} = \{ 2 \}$, so the reverse inclusion is false.

\item True.

  Both $A \cap B$ and $A - B$ are subsets of $A$, so their union is, too, and we have $(A \cap B) \cup (A - B) \subset A$.
  Conversely, let $x \in A$. If we also have $x \in B$, then $x \in A \cap B$. Otherwise, we have $x \in A$ and $x \notin B$, so that $x \in (A - B)$.
  It follows that $A \subset (A \cap B) \cup (A - B)$. With both inclusions, we conclude that $(A \cap B) \cup (A - B) = A$.

\item True.

  Let $(x, y) \in A \times B$. Then we have $x \in A$ and $y \in B$. From $x \in A$ and $A \subset C$ we deduce $x \in C$.
  From $y \in B$ and $B \subset D$ we deduce $y \in D$. From both, we deduce that $(x, y) \in C \times D$, so that $A \times B \subset C \times D$.

\item False.

  The converse of proposition (i) is $A \times B \subset C \times D \implies A \subset C$ and $B \subset D$.
  Take $A = \{ a, b \}, B = \varnothing, C = \{ a \}, D = \{ 1 \}$. Then $A \times B = \varnothing$, $C \times D = \{ (a, 1) \}$, so that $A \times B \subset C \times D$.
  However, $A \not\subset C$.

\item True.

  Suppose $A \times B \subset C \times D$, and that neither $A$ nor $C$ are empty. Let $(x, y) \in A \times B$. Then $x \in A$ and $y \in B$.
  From $A \times B \subset C \times D$ we deduce that $(x, y) \in C \times D$, so that $x \in C$ and $y \in D$. Summing up, we have
  $x \in A \implies x \in C$ and $y \in B \implies y \in D$, which is the definition of $A \subset C$ and $B \subset D$.

\item False.

  Since $A \subset A \cup C$ and $B \subset B \cup D$, we have $(A \times B) \cup (C \times D) \subset (A \cup C) \times (B \cup D)$
  Take $A = \{ a \}, B = \{ 1 \}, C = \{ a, b \}, D = \{ 1, 2 \}$. Then $(b, 1) \in (A \cup C) \times (B \cup D)$, but it is not an element of $(A \times C) \cup (B \times D)$.

\item True.

  Note that if any of the sets $A$, $B$, $C$, or $D$ is empty, then the formula reduces to $\varnothing = \varnothing$, which is trivially true.
  Let $(x, y) \in (A \times B) \cap (C \times D)$. From $(x, y) \in A \times B$, we deduce $x \in A$ and $y \in B$. From $(x, y) \in C \times D$ we deduce $x \in C$ and $y \in D$.
  From both, we deduce $x \in A \cap C$ and $y \in B \cap D$, so that $(A \times B) \cap (C \times D) \subset (A \cap C) \times (B \times D)$.

  Conversely, let $(x, y) \in (A \cap C) \times (B \cap D)$. We have $x \in A \cap C$ and $y \in B \cap D$. From $A \cap C \subset A$ and $B \cap D \subset B$, we deduce $x \in A$ and $y \in C$,
  so that $(x, y) \in A \times B$. From $A \cap C \subset C$ and $B \cap D \subset D$, we deduce $x \in C$ and $y \in D$, so that $(x, y) \in C \times D$.
  From both, we deduce $(A \cap C) \times (B \cap D) \subset (A \times B) \cap (C \times D)$.

\item True.

  If $A = \varnothing$ or $B = \varnothing$, the left-hand side is $\varnothing$, and the right-hand side translates to $\varnothing - D$ for some $D \subset X$.
  This expression is again equal to $\varnothing$ since $\forall x \in X,\; x \notin \varnothing$. So the equality is true.

  If $C = \varnothing$, then $A \times C = \varnothing$ and $B - C = B$, so the equality is again trivially true.

  Otherwise, let $x \in A \times (B - C)$. This is equivalent to having $x \in A$ and $x \in B$ and $x \notin C$, and since $x \in A$ is again equivalent to ($x \in A$ and $x \in A$), we have,
  by commutativity of ``and'', ($x \in A$ and $x \in B$) and ($x \in A$ and $x \notin C$). So the equality is true.

\item True.

  Let $(x, y) \in (A \times C - B \times C) - A \times D$. We have $(x, y) \in (A \times C - B \times C)$ and $(x, y) \notin A \times D$.
  $(x, y) \notin A \times D$ gives ($x \notin A$ or $y \notin D$).
  $(x, y) \in (A \times C - B \times C)$ gives $x \in A$ and $x \notin B$ and $y \in C$.

  Combining both, we get:
  ($x \notin A$ or $y \notin D$) and $x \in A$ and $x \notin B$ and $y \in C$, which simplifies to
  $y \notin D$ and $x \in A$ and $x \notin B$ and $y \in C$, so that
  $(x, y) \in (A - B) \times (C - D)$

  All the above transformations are equivalences, so $(A - B) \times (C - D) = (A \times C - B \times C) - A \times D$

\item False.

  Take $A = \{ a \}, B = \{ 1, 2 \}, C = \varnothing, D = \{ 1 \}$. Then $A \times B - C \times D = \{ (a, 1), (a, 2) \}$,
  but $(A - C) \times (B - D) = \{ a \} \times \{ 2 \} = \{ (a, 2) \}$.

  Let $(x, y) \in (A - C) \times (B - D)$. We have $x \in A$ and $x \notin C$ and $y \in B$ and $y \notin D$, from which we deduce
  $x \in A$ and $y \in B$ and $x \notin C$ and $y \notin D$. From $x \notin C$ and $y \notin D$, we deduce $(x, y) \notin C \times D$.
  Putting both parts together we get $(A - C) \times (B - D) \subset (A \times B) - (C \times D)$.

\end{enumerate}
\end{proof}

\end{document}
