\documentclass[11pt,a4paper,twoside]{article}
\usepackage{mathtools}
\usepackage{amsfonts}
\usepackage{amssymb}
\usepackage{amsthm}
\usepackage{mathrsfs}
\usepackage[shortlabels]{enumitem}

\theoremstyle{definition}
\newcounter{excounter}
\setcounter{excounter}{3}
\newtheorem{exercise}[excounter]{Exercise}

\begin{document}

\begin{exercise}
  Let $f : A \to B$ and $g : B \to C$.
  \begin{enumerate}[(a)]
    \item If $C_0 \subset C$, show that $\left(  g \circ f \right)^{-1} (C_0) = f^{-1} \circ g^{-1} (C_0)$
    \item If $f$ and $g$ are injective, show that $g \circ f$ is injective
    \item If $g \circ f$ is injective, what can you say about the injectivity of $f$ and $g$?
    \item If $f$ and $g$ are surjective, show that $g \circ f$ is surjective
    \item If $g \circ f$ is surjective, what can you say about the surjectivity of $f$ and $g$?
  \end{enumerate}
\end{exercise}

\begin{proof}\hfill
  \begin{enumerate}[(a)]
  \item Let $x \in \left( g \circ f \right)^{-1} (C_0)$. Then $g \circ f (x) \in C_0$, from which we deduce that $f (x) \in g^{-1} (C_0)$,
    and then that $x \in f^{-1} \circ g^{-1} (C_0)$. Thus we have $(g \circ f)^{-1} (C_0) \subset f^{-1} \circ g^{-1} (C_0)$.

    Conversely, let $x \in f^{-1} \circ g^{-1} (C_0) = \left\{ x \in A \mid f (x) \in g^{-1} (C_0) \right\} = f^{-1} \left( g^{-1} \left( C_0 \right) \right)$.
    Then we have $f (x) \in g^{-1} (C_0) = \left\{ y \in B \mid g (y) \in C_0 \right\}$, so that $g \circ f (x) \in C_0$, from which we deduce that
    $f^{-1} \circ g^{-1} (C_0) = \left( g \circ f \right)^{-1} (C_0)$.

  \item Let $x$ and $y$ in $A$ such that $g \circ f (x) = g \circ f (y)$. By injectivity of $g$, we deduce that $f (x) = f (y)$, and by injectivity of $f$, that
    $x = y$. Thus $g \circ f$ is injective.

  \item Let $x$ and $y$ in $A$  such that $f (x) = f (y)$. $g$ being a function, we deduce $g \circ f (x) = g \circ f (y)$. By injectivity of $g \circ f$,
    we deduce that $x = y$, hence that $f$ is injective. Note that $g$ is not necessarily injective: take $A = \left\{ 1, 2 \right\}$, $B = \left\{ 3, 4, 5 \right\}$, $C = \left\{ 6, 7 \right\}$,
    $f (1) = 3$, $f (2) = 4$, $g (3) = 6$, $g (4) = 7$, $g (5) = 7$. Then $g \circ f$ is injective but $g$ itself is not.

  \item Let $z \in C$. By surjectivity of $g$, there exists $y \in B$ such that $z = g (y)$. By surjectivity of $f$, there exists $x \in A$ such that $y = f (x)$.
    Composing both, we get $z = g \circ f (x)$, so $g \circ f$ is surjective.

  \item Let $z \in C$. By surjectivity of $g \circ f$, there exists $x \in A$ such that $z = g \circ f (x) = g \left( f (x) \right)$.
    So $f (x)$ is an antecedent of $z$ by $g$, from which we conclude that $g$ is surjective. $f$ does not need to be surjective, however:
    take $A = \left\{ 1, 2 \right\}$, $B = \left\{ 3, 4, 5 \right\}$, $C = \left\{ 6, 7 \right\}$, $f (1) = 3$, $f (2) = 4$, $g (3) = 6$, $g (4) = 7$, $g (5) = 7$.
    Then $g \circ f$ is injective, but $f$ is not\qedhere

  \end{enumerate}
\end{proof}

\end{document}
