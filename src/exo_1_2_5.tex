\documentclass[11pt,a4paper,twoside]{article}
\usepackage{mathtools}
\usepackage{amsfonts}
\usepackage{amssymb}
\usepackage{amsthm}
\usepackage{mathrsfs}
\usepackage[shortlabels]{enumitem}

\theoremstyle{definition}
\newcounter{excounter}
\setcounter{excounter}{4}
\newtheorem{exercise}[excounter]{Exercise}

\begin{document}

\begin{exercise}
  In general, let us denote the \emph{identity function} for a set $C$ by $i_C$. That is,
  define $i_C : C \to C$ to be the function given by the rule $i_C (x) = x$ for all $x \in C$.
  Given $f : A \to B$, we say that a function $g : B \to A$ is a \emph{left inverse} for $f$ if $g \circ f = i_A$;
  and we say that $h : B \to A$ is a \emph{right inverse} for $f$ if $f \circ h = i_B$.
  \begin{enumerate}[(a)]
    \item Show that if $f$ has a left inverse, $f$ is injective; and if $f$ has a right inverse, $f$ is surjective
    \item Give an example of a function that has a left inverse but no right inverse
    \item Give an example of a function that has a right inverse but no left inverse
    \item Can a function have more than one left inverse? More than one right inverse?
    \item Show that if $f$ has both a left inverse $g$ and a right inverse $h$, then $f$ is bijective and $g = h = f^{-1}$
  \end{enumerate}

\end{exercise}

\begin{proof}\hfill
  \begin{enumerate}[(a)]

  \item If $f$ has a left inverse $g$, then $g \circ f = i_A$. The function $i_A$ being injective, we deduce that $f$ is injective.
    Similarly, if $f$ has a right inverse $h$, then $f \circ h = i_B$. The function $i_B$ being surjective, we deduce that $f$ is surjective.

  \item Let $A = \left\{ 1, 2 \right\}$, $B = \left\{ a, b, c \right\}$, $f (1) = a$, $f (2) = b$. The function $g : B \to A$ defined by
    $g (a) = 1$, $g (b) = 2$, $g (c) = 1$ is a left inverse for $f$. However, $f$ not being surjective implies that it does not have a right inverse.

  \item Consider $A$, $B$, $f$, $g$ as in the exemple defined above. Then $f$ is a right inverse for $g$. Similarly, $g$ not being injective implies it does not have a left inverse.

  \item Left and right inverses are not unique. Still with the definitions from the same example, let $h : B \to A$ defined by $h (a) = 1$, $h (b) = 2$, $h (c) = 2$.
    Then both $g$ and $h$ are left inverses for $f$. Define also $k : A \to B$ by $k (1) = a$, $k (2) = c$. Then both $f$ and $k$ are left inverses for $g$.

  \item Let $f$, $g$ and $h$ be such that $g \circ f = i_A$ and $f \circ h = i_B$.
    Let $y \in B$; we have $y = f \circ h (y)$, so $g (y) = \left( g \circ f \right) \circ h (y) = h (y)$, so $g$ and $h$ are equal.
    Since $f \circ h = i_B$ is surjective, then $f$ is surjective. Since $g \circ f = i_A$ is injective, then $f$ is injective.
    Since $f$ is both surjective and injective, it is bijective. From $f \circ h = i_B = f \circ f^{-1}$, we deduce that $f^{-1} = h$.
    And from $h = g$, we deduce $g = h = f^{-1}$\qedhere

  \end{enumerate}

\end{proof}

\end{document}
