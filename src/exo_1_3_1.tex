\documentclass[11pt,a4paper,twoside]{article}
\usepackage{mathtools}
\usepackage{amsfonts}
\usepackage{amssymb}
\usepackage{amsthm}
\usepackage{mathrsfs}
\usepackage[shortlabels]{enumitem}

\theoremstyle{definition}
\newcounter{excounter}
\setcounter{excounter}{0}
\newtheorem{exercise}[excounter]{Exercise}

\begin{document}

\begin{exercise}

  Define two points $(x_0, y_0)$ and $(x_1, y_1)$ of the plane to be equivalent if $y_0 - x_0^2 = y_1 - x_1^2$.
  Check that this is an equivalence relation and describe the equivalence classes.

\end{exercise}

\begin{proof}\hfill

  Let us note this relation $\mathscr{R}$.
  The reflexivity, symmetry, and transitivity of equality imply that $\mathscr{R}$ also has these properties.
  Thus $\mathscr{R}$ is an equivalence relation.

  For all $C \in \mathbb{R}$, $y - x^2 = C$ is a parabola with focus $(0, C + 1 / 4)$ and directrix $y = C - 1 / 4$.
  So the equivalence classes are all parabolas with $C$ in the above equations varying over $\mathbb{R}$.

\end{proof}

\end{document}
