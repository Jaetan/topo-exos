\documentclass[11pt,a4paper,twoside]{article}
\usepackage{mathtools}
\usepackage{amsfonts}
\usepackage{amssymb}
\usepackage{amsthm}
\usepackage{mathrsfs}
\usepackage[shortlabels]{enumitem}

\theoremstyle{definition}
\newcounter{excounter}
\setcounter{excounter}{9}
\newtheorem{exercise}[excounter]{Exercise}

\begin{document}

\begin{exercise}

  \begin{enumerate}[(a)]

  \item Show that the map $f : ({-1}, 1) \to \mathbb{R}, x \mapsto x / (1 - x^2)$ is order-preserving.
  \item Show that the equation $g (y) = 2y \left. / \left( 1 + \left( 1 + 4y^2 \right)^{1 / 2} \right) \right.$ defines a function $g : \mathbb{R} \to ({-1}, 1)$
    that is both a left and a right inverse for $f$.

  \end{enumerate}

\end{exercise}

\begin{proof}\hfill

  \begin{enumerate}[(a)]

  \item The function $f$ is strictly increasing in the interval $({-1}, 1)$ and thus order-preserving.
    Remark that $f (x) = 1 / 2 \tan \theta$ for $x = \tan ( \theta / 2 )$ and $\theta \in ({-\pi} / 2, \pi / 2)$,
    so that $f$ is the composition of strictly increasing functions, and thus strictly increasing.

  \item As a continuous and strictly increasing function on its domain $({-1}, 1)$, $f$ is bijective and has thus a unique inverse
    (which is both a left and a right inverse). It is thus enough to verify that $f \circ g (y) = y$ for all $y$ in $\mathbb{R}$.
    Letting $t = \sqrt{1 + 4 y^2}$, we get
    \begin{align*}
      1 - g^2 (y) &= 1 - \frac{t^2 - 1}{(1 + t)^2} \\
      &= 1 - \frac{t - 1}{t + 1} = \frac{2}{t + 1} \\
      \frac{g (y)}{1 - g^2 (y)} &= \frac{t + 1}{2} \sqrt{\frac{t - 1}{t + 1}} \\
      &= \sqrt{\frac{t^2 - 1}{4}} = y
    \end{align*}
    for $y \geq 0$. Since $f (-x) = - f (x)$, $g (-y) = - g (y)$, and $y \geq 0 \iff g (y) \geq 0$,
    we apply the above to $-y$ when $y < 0$ to get
    \begin{align*}
      f \circ g (-y) &= f \left( - g (y) \right) = - f \circ g (y) \\
      &= - y
    \end{align*}
    which gives the expected result.

  \end{enumerate}

\end{proof}

\end{document}
