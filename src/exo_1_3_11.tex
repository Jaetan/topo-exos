\documentclass[11pt,a4paper,twoside]{article}
\usepackage{mathtools}
\usepackage{amsfonts}
\usepackage{amssymb}
\usepackage{amsthm}
\usepackage{mathrsfs}
\usepackage[shortlabels]{enumitem}

\theoremstyle{definition}
\newcounter{excounter}
\setcounter{excounter}{10}
\newtheorem{exercise}[excounter]{Exercise}

\begin{document}

\begin{exercise}

  Show that an element in an ordered set has at most one immediate successor and at most one immediate predecessor.
  Show that a subset of an ordered set has at most one smallest element and at most one largest element.

\end{exercise}

\begin{proof}

  Let $A$ be an ordered set. If $A = \varnothing$, the results above are vacuously true, so we suppose $A$ is not empty.

  Let $x \in A$ and suppose $b_0$ and $b_1$ are immediate successors of $x$.
  If $b_0 \neq b_1$, then either $b_0 < b_1$ or $b_1 < b_0$. These cases are symmetric so we can assume for example that $b_0 < b_1$.
  Then $b_0 \in \{ y \in A \mid x < y < b_1\}$, so that $b_1$ cannot be the immediate successor of $x$.
  From the above we deduce that $b_0 = b_1$, so that the immediate successor is unique if it exists.

  A similar proof shows that the immediate predecessor of an element is unique if it exists.

  Let $A$ be an ordered set, $B \subset A$ and $s_0$ and $s_1$ two smallest elements of $B$ in $A$. Then we have $s_0 \in B$ and $s_1 \in B$.
  Since $s_0$ (resp. $s_1$) is the smallest element of $B$ in $A$, we must have $s_0 \leq s_1$ (resp. $s_1 \leq s_0$).
  If $s_0 < s_1$, then we cannot have $s_1 \leq s_0$; and similarly, $s_1 < s_0$ is also impossible. By comparability of the order relation on $A$,
  if $s_0 \neq s_1$ then one of $s_0 < s_1$ or $s_1 < s_0$ must be true, though. We deduce that we cannot have $s_0 \neq s_1$ either,
  and therefore the smallest element of $B$ must be unique.

  A similar proof shows that the largest element of a subset of an ordered set is unique.

\end{proof}

\end{document}
