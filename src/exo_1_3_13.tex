\documentclass[11pt,a4paper,twoside]{article}
\usepackage{mathtools}
\usepackage{amsfonts}
\usepackage{amssymb}
\usepackage{amsthm}
\usepackage{mathrsfs}
\usepackage[shortlabels]{enumitem}

\theoremstyle{definition}
\newcounter{excounter}
\setcounter{excounter}{12}
\newtheorem{exercise}[excounter]{Exercise}

\begin{document}

\begin{exercise}

  Prove the following:\\
  \emph{Theorem: if an ordered set $A$ has the least upper bound property, then it has the greatest lower bound property.}

\end{exercise}

\begin{proof}

  Let $A$ be an ordered set. If $A = \varnothing$, then there is no nonempty subset of $A$, so the theorem is vacuously true.
  Suppose then that $A \neq \varnothing$, and let $C$ be a nonempty subset of $A$ that is bounded below by $c \in A$.
  Let $B$ be the set of lower bounds of $C$; we have $c \in B$ so $B$ is nonempty.
  Since $C$ is nonempty, we can take $c_0 \in C$, and we have $\forall y \in B, \quad y \leq c_0$, so that $B$ is bounded above.
  Since $A$ has the least upper bound property, $B$ has a lowest upper bound $u$.
  \\

  By definition, $u$ is the smallest element of the set of upper bounds of $B$ in $A$. If there exists $c_1 \in C$ such that $c_1 < u$,
  then let $b \in B$. By definition of $B$, we have $b \leq c_1$, so that $c_1$ is an upper bound of $B$ that is smaller than $u$.
  This is a contradiction, so $\forall c \in C, \quad u \leq c$, and $u$ is a lower bound of $C$.
  \\

  If $v \in A$ is a lower bound of $C$, then by definition of $B$, $v \in B$, so that $v \leq u$.
  From this we deduce that $u$ is the greatest lower bound of $C$ in $A$, so that $A$ has the greatest lower bound property.\qedhere

\end{proof}

\end{document}
