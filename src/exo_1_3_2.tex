\documentclass[11pt,a4paper,twoside]{article}
\usepackage{mathtools}
\usepackage{amsfonts}
\usepackage{amssymb}
\usepackage{amsthm}
\usepackage{mathrsfs}
\usepackage[shortlabels]{enumitem}

\theoremstyle{definition}
\newcounter{excounter}
\setcounter{excounter}{1}
\newtheorem{exercise}[excounter]{Exercise}

\begin{document}

\begin{exercise}

  Let $C$ be a relation on a set $A$. If $A_0 \subset A$, define the \emph{restriction} of $C$ to $A_0$ to be the relation $C \cap (A_0 \times A_0)$.
  Show that the restriction of an equivalence relation is an equivalence relation.

\end{exercise}

\begin{proof}\hfill

  Let $A_0 \subset A$ and $x, y, z \in A_0$. The reflexivity, symmetry and transitivity of the restriction of $C$ to $A_0$ come from considering
  $x, y, z$ as elements of $A$, applying the corresponding property of $C$, and remarking that all the elements involved were in $A_0$. This implies
  that the properties already hold in $A_0$.

\end{proof}

\end{document}
