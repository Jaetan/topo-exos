\documentclass[11pt,a4paper,twoside]{article}
\usepackage{mathtools}
\usepackage{amsfonts}
\usepackage{amssymb}
\usepackage{amsthm}
\usepackage{mathrsfs}
\usepackage[shortlabels]{enumitem}

\theoremstyle{definition}
\newcounter{excounter}
\setcounter{excounter}{2}
\newtheorem{exercise}[excounter]{Exercise}

\begin{document}

\begin{exercise}

  Here is a ``proof'' that every relation $C$ that is both symmetric and transitive is also reflexive:
  ``Since $C$ is symmetric, $a C b$ implies $b C a$. Since $C$ is transitive, $a C b$ and $b C a$ together imply $a C a$, as desired.''
  Find the flaw in this argument.

\end{exercise}

\begin{proof}\hfill

  Let $A = \{ 1, 2, 3 \}$ and $C$ the relation defined by the following part of $A \times A$:
  $C = \big\{ (1, 1), (1, 3), (3, 3), (3, 1) \big\}$. By construction, $C$ is both symmetric and transitive, but is not reflexive since $(2, 2) \notin C$.
  The flaw in the reasoning comes from the fact that it only considers elements of $C$, whereas reflexivity is $\forall x \in A,\quad x C x$.

\end{proof}

\end{document}
