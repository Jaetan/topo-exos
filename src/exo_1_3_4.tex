\documentclass[11pt,a4paper,twoside]{article}
\usepackage{mathtools}
\usepackage{amsfonts}
\usepackage{amssymb}
\usepackage{amsthm}
\usepackage{mathrsfs}
\usepackage[shortlabels]{enumitem}

\theoremstyle{definition}
\newcounter{excounter}
\setcounter{excounter}{3}
\newtheorem{exercise}[excounter]{Exercise}

\begin{document}

\begin{exercise}

  Let $f : A \to B$ be a surjective function. Let us define a relation on $A$ by setting
  $a_0 \sim a_1$ if
  \begin{equation*}
    f (a_0) = f (a_1)
  \end{equation*}
  \begin{enumerate}[(a)]
  \item Show that this is an equivalence relation
  \item Let $A^*$ be the set of equivalence classes. Show there is a bijective correspondence of $A^*$ with $B$
  \end{enumerate}

\end{exercise}

\begin{proof}\hfill

  \begin{enumerate}[(a)]
  \item The fact that every $x \in A$ has a unique image by $f$, and that equality is an equivalence relation, imply that $\sim$ is an equivalence relation on $A$.
  \item Let $x^* \in A^*$. By definition of $\sim$, for all $y, z \in x^*$, we have $f (y) = f (z)$. So the following definition uniquely associates an image to an element,
    and is thus a function:
    \begin{align*}
      \phi : A^* &\to B \\
      x^* &\mapsto f (x)
    \end{align*}
    The function $\phi$ is surjective: $f$ being surjective, $\forall y \in B, \quad \exists z \in A, \quad f (x) = y$. Since $\sim$ is an equivalence relation on $A$,
   $x \sim x$, from which we deduce that $\phi (x^*) = y$, which is the surjectivity of $\phi$.
  \end{enumerate}

\end{proof}

\end{document}
