\documentclass[11pt,a4paper,twoside]{article}
\usepackage{mathtools}
\usepackage{amsfonts}
\usepackage{amssymb}
\usepackage{amsthm}
\usepackage{mathrsfs}
\usepackage[shortlabels]{enumitem}

\theoremstyle{definition}
\newcounter{excounter}
\setcounter{excounter}{7}
\newtheorem{exercise}[excounter]{Exercise}

\begin{document}

\begin{exercise}

  Show that the relation defined in Example 7 is an order relation.
  For reference, Example 7 is reproduced below.

  Define $x C y$ if $x^2 < y^2$, or if $x^2 = y^2$ and $x < y$, for $x, y \in \mathbb{R}$.

\end{exercise}

\begin{proof}\hfill

  \begin{description}

  \item [comparability] Let $x, y \in \mathbb{R}$ such that $x \neq y$.
    Suppose that $x^2 = y^2$. Then either $x < y$ or $y < x$, since they are different, so that $x$ and $y$ are comparable by $C$.
    Otherwise, $x^2 \neq y^2$; the comparability of the real numbers $x^2$ and $y^2$ for the usual order relation $<$ on $\mathbb{R}$ gives us
    either $x^2 < y^2$ or $y^2 < x^2$. From this we deduce that either $x C y$ or $y C x$, so that $x$ and $y$ are comparable by $C$.

  \item [non-reflexivity] Let $x \in \mathbb{R}$. The non-reflexivity of $<$ implies that we have neither $x^2 < x^2$ nor $x < x$.
    From this we deduce that we do not have $x C x$ either.

  \item [transitivity] Let $x, y, z \in \mathbb{R}$ such that $x C y$ and $y C z$.
    Suppose $x^2 < y^2$. Then if we also have $y^2 < z^2$, we deduce that $x C z$ by transitivity of $<$ on $\mathbb{R}$.
    Otherwise, $y^2 = z^2$ and $y < z$. Then we have $x^2 < z^2$ so that $x C z$.

    Suppose that $x^2 = y^2$ and $x < y$. If $y^2 < z^2$, then $x^2 < z^2$ and thus $x C z$.
    Otherwise $x^2 = y^2 = z^2$ and $x < y < z$, from which we deduce again $x C z$

  \end{description}

  From the above we conclude that $C$ is an order relation on $\mathbb{R}$.

\end{proof}

\end{document}
