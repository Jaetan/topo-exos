\documentclass[11pt,a4paper,twoside]{article}
\usepackage{mathtools}
\usepackage{amsfonts}
\usepackage{amssymb}
\usepackage{amsthm}
\usepackage{mathrsfs}
\usepackage[shortlabels]{enumitem}

\theoremstyle{definition}
\newcounter{excounter}
\setcounter{excounter}{10}
\newtheorem{exercise}[excounter]{Exercise}

\begin{document}

\begin{exercise}

  Given $m \in \mathbb{Z}$, we say that $m$ is \emph{even} if $m / 2 \in \mathbb{Z}$, and $m$ is \emph{odd} otherwise.

  \begin{enumerate}[(a)]

  \item Show that if $m$ is odd, $m = 2 n + 1$ for some $n \in \mathbb{Z}$. [\emph{Hint:} choose $n$ such that $n < m / 2 < n + 1$.]
  \item Show that if $p$ and $q$ are odd, so are $p \cdot q$ and $p^n$, for any $n \in \mathbb{Z}_+$.
  \item Show that if $a > 0$ is rational, then $a = m / n$ for some $m, n \in \mathbb{Z}_+$ where not both $m$ and $n$ are even.
    [\emph{Hint:} Let $n$ be the smallest element of the set $\{ x \mid x \in \mathbb{Z}_+$ and $x \cdot a \in \mathbb{Z}_+ \}$.]
  \item \emph{Theorem: }$\sqrt{2}$ \emph{is irrational.}

  \end{enumerate}

\end{exercise}

\begin{proof}\hfill

  \begin{enumerate}[(a)]

  \item Let $m \in \mathbb{Z}$ be an odd number; then $m / 2 \notin \mathbb{Z}$, so that by exercise 1.4.9-b) there exists a unique $n \in \mathbb{Z}$ such that
    $n < m / 2 < n + 1$. This implies that $2 n < m < 2 n + 2$. The only integer verifying this inequality is $2 n + 1$, so $m = 2 n + 1$.

  \item Let $p, q \in \mathbb{Z}$ be odd numbers. From the previous point, there exist $a, b \in \mathbb{Z}$ such that $p = 2 a + 1$ and $q = 2 b + 1$.
    This gives us $p \cdot q = (2 a + 1) (2 b + 1) = 2 \left( a \left( 2 b + 1 \right) + b \right) + 1$, so that $p \cdot q = 2 m + 1$ for some $m \in \mathbb{Z}$, and thus is odd.

    Let $A = \{ n \in \mathbb{Z} \mid p^n$ is odd$\,\}$. Since $p$ is odd, we have $1 \in A$.
    Suppose that $n \in \mathbb{Z}_+$ such that $n \in A$. Then $p^n$ is odd, and $p^{n + 1} = p \cdot p^n$ is the product of two odd numbers.
    Therefore $p^{n + 1}$ is odd, so that $n + 1 \in A$ and $A$ is inductive. We conclude that $A = \mathbb{Z}_+$, which is the expected result.

  \item Let $a > 0$ be a rational number and $A = \{ x \in \mathbb{Z}_+ \mid x \cdot a \in \mathbb{Z}_+ \}$. There exist 2 integers $p, q$ with $q \neq 0$ such that $a = p / q$.
    We have $q \cdot a = p$ and $a > 0$, so that $p$ and $q$ are either both positive, or both negative. Let us further suppose that $p > 0$ and $q > 0$.
    Then $q \cdot a \in A$, so that $A \neq \varnothing$. Therefore $A$ has a smallest element $m \in \mathbb{Z}_+$.
    We have $m \cdot a = n \in \mathbb{Z}_+$. If both $m$ and $n$ are even, then $(m / 2) \cdot a = n / 2 \in \mathbb{Z}_+$, so that $m / 2 \in A$.
    Since $m > 0$, this leads to $m / 2 < m$, which contradicts the fact that $m$ is the smallest element of $A$.
    Therefore $a = n / m$ where not both $m$ and $n$ are even.

  \item Suppose that $\sqrt{2} = n / m$ with $n$ and $m$ not both even. Then $2 m^2 = n^2$, so that $n^2$ is even.
    If $n$ were odd, then $n^2$ would also be odd; so $n$ is even, and there exists $p \in \mathbb{Z}$ such that $n = 2 p$.
    From this we deduce that $4 p^2 = 2 m^2$, so that $2 p^2 = m^2$. Therefore $m^2$ is even, and so is $m$.
    This is a contradiction with $n$ and $m$ not being both even, so $\sqrt{2}$ is irrational.\qedhere

  \end{enumerate}

\end{proof}

\end{document}
