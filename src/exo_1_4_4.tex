\documentclass[11pt,a4paper,twoside]{article}
\usepackage{mathtools}
\usepackage{amsfonts}
\usepackage{amssymb}
\usepackage{amsthm}
\usepackage{mathrsfs}
\usepackage[shortlabels]{enumitem}

\theoremstyle{definition}
\newcounter{excounter}
\setcounter{excounter}{3}
\newtheorem{exercise}[excounter]{Exercise}

\begin{document}

\begin{exercise}\hfill

  \begin{enumerate}[(a)]

  \item Prove by induction that given $n \in \mathbb{Z}_+$, every nonempty subset of $\{ 1, 2, \dotsc, n \}$ has a largest element
  \item Explain why you cannot conclude from (a) that every nonempty subset of $\mathbb{Z}_+$ has a largest element

  \end{enumerate}

\end{exercise}

\begin{proof}\hfill

  \begin{itemize}[(a)]

  \item Let $A$ be the subset of $\mathbb{Z}_+$ such that $\forall n \in A$, every nonempty subset of $\{ 1, 2, \dotsc, n \}$ has a largest element.
    Then $1 \in A$ since the only nonempty subset of $\{ 1 \}$ is itself, and $1$ is thus its largest element. Suppose that $n \in A$, and consider
    a nonempty subset $A_0$ of $\{ 1, 2, \dotsc, n + 1 \}$. If $n + 1 \in A_0$, then it is the largest element of $A_0$.
    Otherwise, $A_0 \cap \{ 1, 2, \dotsc, n \}$ is nonempty and $n \in A$, so that $A_0$ has a largest element.
    By induction we deduce that $A = \mathbb{Z}_+$.

  \item The previous point showed that every nonempty subset of $\mathbb{Z}_+$ that has an upper bound has a largest element.
    There are nonempty subsets of $\mathbb{Z}_+$ that do not have an upper bound, for example $\mathbb{Z}_+$ itself, so the result would not hold
    for any nonempty subset of $\mathbb{Z}_+$. \qedhere

  \end{itemize}

\end{proof}

\end{document}
