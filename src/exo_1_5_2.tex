\documentclass[11pt,a4paper,twoside]{article}
\usepackage{mathtools}
\usepackage{amsfonts}
\usepackage{amssymb}
\usepackage{amsthm}
\usepackage{mathrsfs}
\usepackage[shortlabels]{enumitem}

\theoremstyle{definition}
\newcounter{excounter}
\setcounter{excounter}{1}
\newtheorem{exercise}[excounter]{Exercise}

\begin{document}

\begin{exercise}\hfill

  \begin{enumerate}[(a)]
  \item Show that if $n > 1$ there is a bijective correspondence of
    \begin{align*}
      A_1 \times A_2 \times \dotsb \times A_n &&\text{with}&& \left( A_1 \times A_2 \times \dotsb \times A_{n - 1} \right) \times A_n
    \end{align*}
  \item Given the indexed family $\big\{ A_1, A_2, \dotsc \big\}$, let $B_i = A_{2 i - 1} \times A_{2 i}$ for each positive integer $i$.
    Show that there is a bijective correspondence of $A_1 \times A_2 \times \dotsb$ with $B_1 \times B_2 \times \dotsb$.
  \end{enumerate}

\end{exercise}

\begin{proof}\hfill

  \begin{enumerate}[(a)]

  \item Let $n > 1$. If for some $i \leq n$, $A_i = \varnothing$, then $A_1 \times \dotsb \times A_n = \varnothing$.
    If $i < n$, then $A_1 \times \dotsb \times A_{n - 1}$ is also empty, which implies that $\left( A_1 \times \dotsb \times A_{n - 1} \right) \times A_n$ is empty.
    If $i = n$, then $\left ( A_1 \times \dotsb \times A_{n - 1} \right) \times A_n = \left( A_1 \times \dotsb \times A_{n - 1} \right) \times \varnothing = \varnothing$.

    Let us then suppose that $\forall i, \; 1 \leq i \leq n, \; A_i \neq \varnothing$. Let
    \begin{align*}
      p_1 : A_1 \times \dotso \times A_{n - 1} \times A_n &\to A_1 \times \dotsb \times A_{n - 1} \\
      (x_1, \dotsc, x_{n - 1}, x_n) &\mapsto (x_1, \dotsc, x_{n - 1}) \\
      p_2 : A_1 \times \dotso \times A_{n - 1} \times A_n &\to A_n \\
      (x_1, \dotsc, x_{n - 1}, x_n) &\mapsto x_n
    \end{align*}
    and define
    \begin{align*}
      q : A_1 \times \dotso \times A_n &\to (A_1 \times \dotso \times A_{n - 1}) \times A_n \\
      x &\mapsto \big( p_1 (x), p_2 (x) \big)
    \end{align*}

    Let $x, y \in A_1 \times \dotso \times A_n$. If
    \begin{align*}
      q (x) &= q (y) \quad\text{then} \\
      \big( p_1 (x), p_2 (x) \big) &= \big( p_1 (y), p_2 (y) \big)
    \end{align*}
    so $p_1 (x) = p_1 (y)$ and $p_2 (x) = p_2 (y)$ by definition of the cartesian product of 2 sets.
    From $p_1 (x) = p_1 (y)$, we deduce that the first $n - 1$ coordinates of $x$ and $y$ are equal, and from $p_2 (x) = p_2 (y)$
    we deduce that their $n$-th coordinates are also equal; so $x = y$ and $q$ is injective.

    Let
    \begin{equation*}
      x = \big( (x_1, \dotsc, x_{n - 1}), x_n \big) \in ( A_1 \times \dotso \times A_{n - 1} ) \times A_n
    \end{equation*}
    Take $y = (x_1, \dotsc, x_n)$; we have
    \begin{align*}
      p_1 (y) &= (x_1, \dotsc, x_{n - 1}) \quad\text{and} \\
      p_2 (y) &= x_n
    \end{align*}
    so that $q (y) = x$ and $q$ is surjective.
    Therefore $A_1 \times \dotso \times A_{n - 1} \times A_n$ and $( A_1 \times \dotso \times A_{n - 1} ) \times A_n$ are in bijective correspondence to each other.
    This result justifies reasoning by induction on the number of sets in a cartesian product with index set a subset of $\mathbb{Z}_+$.

  \item For $A$ and $B$ nonempty sets, define
    \begin{align*}
      \pi_1 : A \times B &\to A &\pi_2 : A \times B &\to B \\
      (x, y) &\mapsto x &(x, y) &\mapsto y
    \end{align*}
    We will use such functions for $A = A_{2 i - 1}$ and $B = A_{2 i}$, for all $i \in \mathbb{Z}_+$. Since the domain and range are different for each $i$,
    we get two families of distinct functions $( \pi_{1, i} )$ and $( \pi_{2, i} )$ for $i \in \mathbb{Z}_+$. In order to keep notations simple in what follows,
    we will abusively write $\pi_1$ (respectively $\pi_2$) when we actually mean $\pi_{1, i}$ (respectively $\pi_{2, i}$) for some $i$.
    Let
    \begin{align*}
      \phi : B_1 \times B_2 \times \dotsb &\to A_1 \times A_2 \times \dotsb \\
      ( x_1, x_2, \dotsc ) &\mapsto \big( \pi_1 (x_1), \pi_2 (x_1), \pi_1 (x_2), \pi_2 (x_2), \dotsc \big)
    \end{align*}
    and
    \begin{align*}
      \psi : A_1 \times A_2 \times \dotsb &\to B_1 \times B_2 \times \dotsb \\
      (x_1, x_2, \dotsc) &\mapsto (y_1, y_2, \dotsc) \quad\text{ where }\quad y_i = ( x_{2 i - 1}, x_{2 i} )
    \end{align*}
    For all $i \in \mathbb{Z}_+$, $y_i = ( x_{2 i - 1}, x_{2 i} ) = \big( \pi_1 (y_i), \pi_2 (y_i) \big))$, so
    \begin{align*}
      \phi \circ \psi = i_A \quad\text{ with }\quad A = A_1 \times A_2 \times \dotsb \\
      \psi \circ \phi = i_B \quad\text{ with }\quad B = B_1 \times B_2 \times \dotsb
    \end{align*}
    So $\phi$ and $\psi$ are bijective and inverse of each other. From this we deduce that $A_1 \times A_2 \times \dotsb$ and
    $( A_1 \times A_2 ) \times ( A_3 \times A_4 ) \times \dotsb$ are in bijective relation to each other.

  \end{enumerate}

\end{proof}

\end{document}
