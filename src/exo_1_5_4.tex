\documentclass[11pt,a4paper,twoside]{article}
\usepackage{mathtools}
\usepackage{amsfonts}
\usepackage{amssymb}
\usepackage{amsthm}
\usepackage{mathrsfs}
\usepackage[shortlabels]{enumitem}

\theoremstyle{definition}
\newcounter{excounter}
\setcounter{excounter}{3}
\newtheorem{exercise}[excounter]{Exercise}

\begin{document}

\begin{exercise}

  Let $m, n \in \mathbb{Z}_+$. Let $X \neq \varnothing$.
  \begin{enumerate}[(a)]
  \item If $m \leq n$, find an injective map $f : X^m \to X^n$.
  \item Find a bijective map $g : X^m \times X^n \to X^{m + n}$.
  \item Find an injective map $h : X^n \to X^\omega$.
  \item Find a bijective map $k : X^n \times X^\omega \to X^\omega$.
  \item Find a bijective map $l : X^\omega \times X^\omega \to X^\omega$.
  \item If $A \subset B$, find an ijective map $m : (A^\omega)^n \to B^\omega$.
  \end{enumerate}

\end{exercise}

\begin{proof}\hfill

  \begin{enumerate}[(a)]

  \item Since $X \neq \varnothing$, let $x_0 \in X$, and define
    \begin{align*}
      f : X^m &\to X^n \\
      (x_1, x_2, \dotsc, x_m) &\mapsto (y_1, y_2, \dotsc, y_n)
    \end{align*}
    with
    \begin{equation*}
      \forall i \in \{ 1, 2, \dotsc, n \}, \; \begin{cases}
        y_i = x_i \quad& \text{ if } \quad 1 \leq i \leq m \\
        y_i = x_0 \quad& \text{ if } \quad m + 1 \leq i \leq n
      \end{cases}
    \end{equation*}
    Let
    \begin{align*}
      (y_1, y_2, \dotsc, y_n) &= f (x_1, x_2, \dotsc x_m) \quad\text{and} \\
      (y'_1, y'_2, \dotsc, y'_n) &= f (x'_1, x'_2, \dotsc, x'_m)
    \end{align*}
    and suppose that $(y_1, y_2, \dotsc, y_n) = (y'_1, y'_2, \dotsc, y'_n)$.
    Then by definition of $f$, $y_i = x_0 = y'_i$ for all $i$ such that $m + 1 \leq i \leq n$,
    and $y_i = x_i = y'_i = x'_i$ for all $i$ such that $1 \leq i \leq m$. From this we deduce that
    $\forall i, \; 1 \leq i \leq m \implies x_i = x'_i$, so that $f$ is injective.

  \item Define
    \begin{align*}
      g : X^m \times X^n &\to X^{m + n} \\
      (x, y) &\mapsto (z_1, z_2, \dotsc, z_{m + n})
    \end{align*}
    with
    \begin{equation*}
      \forall i \in \{ 1, 2, \dotsc, m + n \}, \; \begin{cases}
        z_i = x_i \quad& \text{if} \quad 1 \leq i \leq m \\
        z_i = y_i \quad& \text{if} \quad m + 1 \leq i \leq m + n
      \end{cases}
    \end{equation*}
    Let
    \begin{align*}
      z = (z_1, z_2, \dotsc, z_{m + n}) &= g ((x_1, x_2, \dotsc, x_m), (y_1, y_2, \dotsc, y_n)) \\
      z' = (z'_1, z'_2, \dotsc, z'_{m + n}) &= g ((x'_1, x'_2, \dotsc, x'_m), (y'_1, y'_2, \dotsc, y'_n))
    \end{align*}
    and suppose that $z = z'$.
    Then for all $i$ such that $1 \leq i \leq m$, $z_i = x_i = z'_i = x'_i$,
    and for all $i$ such that $m + 1 \leq i \leq m + n$, $z_i = y_i = z'_i = y'_i$,
    from which we deduce that $g$ is injective.
    Let now $z = (z_1, z_2, \dotsc, z_{m + n}) \in X^{m + n}$, and define
    \begin{align*}
      x &= (z_1, z_2, \dotsc, z_m) \\
      y &= (z_{m + 1}, z_{m + 2}, \dotsc, z_{m + n})
    \end{align*}
    Then $(x, y) \in X^m \times X^n$, and $g (x, y) = z$, so that $g$ is surjective.

  \item Let $x_0 \in X$ and define
    \begin{align*}
      h : X^n &\to X^\omega \\
      (x_1, x_2, \dotsc, x_n) &\mapsto (z_1, z_2, \dotsc)
    \end{align*}
    such that
    \begin{align*}
      z_i = x_i &\quad\text{ if } \quad 1 \leq i \leq n \\
      z_i = x_0 &\quad\text{ if } \quad i > n
    \end{align*}
    Let
    \begin{align*}
      z = (z_1, z_2, \dotsc) &= h (x_1, x_2, \dotsc, x_n) \\
      z' = (z'_1, z'_2, \dotsc) &= h (x'_1, x'_2, \dotsc, x'_n)
    \end{align*}
    and suppose that $z = z'$. Then we have
    \begin{equation*}
      \forall i \in \mathbb{Z}_+, \quad \begin{cases}
        z_i = z'_i = x_0 &\quad\text{ if }\quad i > n \\
        z_i = x_i = z'_i = x'_i &\quad\text{ if }\quad 1 \leq i \leq n
      \end{cases}
    \end{equation*}
    so that $h$ is injective.

  \item Let
    \begin{align*}
      k : X^n \times X^\omega &\to X^\omega \\
      (x, y) &\mapsto z
    \end{align*}
    such that
    \begin{equation*}
      \forall i \in \mathbb{Z}_+, \; \begin{cases}
        z_i = x_i &\quad\text{ if }\quad 1 \leq i \leq n \\
        z_i = y_i &\quad\text{ if }\quad i > n
      \end{cases}
    \end{equation*}
    Let
    \begin{align*}
      z &= (z_1, z_2, \dotsc) & x &= (x_1, x_2, \dotsc, x_n) & y &= (y_1, y_2, \dotsc) \\
      z' &= (z'_1, z'_2, \dotsc) & x' &= (x'_1, x'_2, \dotsc, x'_n) & y' &= (y'_1, y'_2, \dotsc)
    \end{align*}
    and suppose that $z = k (x, y) = z' = k (x', y')$. Then,
    \begin{equation*}
      \forall i \in \mathbb{Z}_+, \; \begin{cases}
        z_i = z'_i = y_i = y'_i &\quad\text{ if }\quad i > n \\
        z_i = z'_i = x_i = x'_i &\quad\text{ if }\quad 1 \leq i \leq n
      \end{cases}
    \end{equation*}
    so that $k$ is injective.
    Next, let $z = (z_1, z_2, \dotsc) \in X^\omega$ and take $x \in X^n$ and $y \in X^\omega$ such that
    \begin{equation*}
      \forall i \in \mathbb{Z}_+, \; \begin{cases}
        x_i = z_i &\quad\text{ if }\quad 1 \leq i \leq n \\
        y_i = z_i &\quad\text{ if }\quad i > n
      \end{cases}
    \end{equation*}
    Then $k (x, y) = z$, so that $k$ is surjective.

  \item Let
    \begin{align*}
      l : X^\omega \times X^\omega &\to X^\omega \\
      (x, y) &\mapsto z
    \end{align*}
    such that
    \begin{equation*}
      \forall i \in \mathbb{Z}_+, \; \begin{cases}
        z_i = x_i &\quad\text{ if $i$ is even } \\
        z_i = y_i &\quad\text{ if $i$ is odd }
      \end{cases}
    \end{equation*}
    Let
    \begin{align*}
      z &= (z_1, z_2, \dotsc) & x &= (x_1, x_2, \dotsc) & y &= (y_1, y_2, \dotsc) \\
      z' &= (z'_1, z'_2, \dotsc) & x' &= (x'_1, x'_2, \dotsc) & y' &= (y'_1, y'_2, \dotsc)
    \end{align*}
    and suppose that $z = l (x, y) = z' = l (x', y')$. Then,
    \begin{equation*}
      \forall i \in \mathbb{Z}, \; \begin{cases}
        z_{2 i} = x_i = z'_{2 i} = x'_i \\
        z_{2 i + 1} = y_i = z'_{2 i + 1} = y'_i
      \end{cases}
    \end{equation*}
    so that for all $i$, $x_i = x'_i$ and $y_i = y'_i$, and $l$ is injective.
    Next, let $z \in X^\omega$, and define $x \in X^\omega$ and $y \in X^\omega$ by
    \begin{align*}
      \forall i \in \mathbb{Z}_+, \; x_i = z_{2 i} \text{ and } y_i = z_{2 i + 1}
    \end{align*}
    Then $x$ and $y$ are elements of $X^\omega$ and $l (x, y) = z$, so that $l$ is surjective.

  \item To define such a function, we need $B^\omega \neq \varnothing$, which implies $B \neq \varnothing$.
    Let
    \begin{align*}
      m : (A^\omega)^n &\to B^\omega \\
      (a_1, a_2, \dotsc, a_n) &\mapsto (a_{1, 1}, a_{1, 2}, \dotsc, a_{1, n}, a_{2, 1}, \dotsc )
    \end{align*}
    noting $a_i = (a_{1, i}, a_{2, i}, \dotsc)$.
    Suppose that $a, a' \in (A^n)^\omega$ are such that $a \neq a'$. Then there exists $r \in \{ 1, 2, \dotsc, n \}$ such that $a_r \neq a'_r$,
    which in turns implies that there exists $q \in \mathbb{Z}_+$ such that $a_{q, r} \neq a'_{q, r}$. The terms
    $a_{q, r}$ and $a'_{q, r}$ appear in $m (a)$ and $m (a')$ at the same index $i$, so $m (a) \neq m (a')$ and $m$ is therefore injective.

  \end{enumerate}

\end{proof}

\end{document}
