\documentclass[11pt,a4paper,twoside]{article}
\usepackage{mathtools}
\usepackage{amsfonts}
\usepackage{amssymb}
\usepackage{amsthm}
\usepackage{mathrsfs}
\usepackage[shortlabels]{enumitem}

\theoremstyle{definition}
\newcounter{excounter}
\setcounter{excounter}{4}
\newtheorem{exercise}[excounter]{Exercise}

\begin{document}

\begin{exercise}

  Which of the following subsets of $\mathbb{R}^\omega$ can be expressed as the cartesian product of subsets of $\mathbb{R}$?
  \begin{enumerate}[(a)]
  \item $\{ \mathbf{x} \mid x_i \text{ is an integer for all } i \}$
  \item $\{ \mathbf{x} \mid x_i \geq i \text{ for all } i \}$
  \item $\{ \mathbf{x} \mid x_i \text{ is an integer for all } i \geq 100 \}$
  \item $\{ \mathbf{x} \mid x_2 = x_3 \}$
  \end{enumerate}

\end{exercise}

\begin{proof}\hfill
  
  \begin{enumerate}[(a)]

  \item Let $A = \{ \mathbf{x} \mid x_i \text{ is an integer for all } i \}$; then for all $i$, $x_i \in \mathbb{Z}_+$ so that $A \subset \mathbb{Z}_+^\omega$.
    Conversely, for $x \in \mathbb{Z}_+^\omega$, $x_i$ is an integer by definition of $\mathbb{Z}_+^\omega$, so $A = \mathbb{Z}_+^\omega$.

  \item Let $A = \{ \mathbf{x} \mid x_i \geq i \text{ for all } i \}$, and let $B = \prod_{i \in \mathbb{Z}_+} [i, {+\infty})$.
    Then for all $i$ and $y \in [i, {+\infty})$, $y \geq i$, so we can take $x_i \in [i, {+\infty})$ to satisfy the condition in $A$. Therefore $B \subset A$.
    Conversely, for all $\mathbf{x} \in A$ and all $i \in \mathbb{Z}_+$, we have $x_i \geq i$, so that $x_i \in [i, {+\infty})$, from which we conclude that $B = A$.

  \item Let $A = \{ \mathbf{x} \mid x_i \text{ is an integer for all } i \geq 100 \}$, and let $B = \mathbb{R}^{99} \times \mathbb{Z}_+^\omega$.
    For all $\mathbf{x} \in B$, $x_i$ is an integer if $i \geq 100$, so $B \subset A$.
    Conversely, for all $\mathbf{x} \in A$, the first $99$ coordinates of $\mathbf{x}$ are only required to be real numbers, so that $\mathbf{x} \in B$, and $B = A$.

  \item Let $A = \{ \mathbf{x} \mid x_2 = x_3 \}$, and suppose there exist $A_1, A_2, \dotsc$ subsets of $\mathbb{R}$ such that $A = A_1 \times A_2 \times \dotsb$.
    For all $\mathbf{x} \in A$, we have $x_2 = x_3$, so that $A_2 = A_3$. Suppose that $A_2$ contains at least 2 elements $a$ and $b$.
    Then the $\omega$-tuple $(x_1, a, b, x_4, \dotsc)$ is an element of $A_1 \times A_2 \times \dotsb$, but is not an element of $A$. So $A_2$ has at most 1 element.
    Suppose there exists $a \in A_2$. Let $\mathbf{x} = ( x_1, a, a, x_4, \dotsc )$ and $\mathbf{y} = (x_1, a + 1, a + 1, x_4, \dotsc)$ for some $x_1, x_4, \dotsc$ elements of $\mathbb{R}$.
    Then both $\mathbf{x}$ and $\mathbf{y}$ are elements of $A$, but $\mathbf{y}$ is not an element of $A_1 \times A_2 \times \dotsb$. So $A_2$ must be empty,
    in which case (because of the axiom of choice), $A_1 \times A_2 \times \dotsb = \varnothing$. Last, note that $(0, 0, 0, \dotsc) \in A$ so $A \neq \varnothing$.
    Therefore, $A \neq A_1 \times A_2 \times \dotsb$ for any subsets $A_1, A_2, \dotsc$ of $\mathbb{R}$.

  \end{enumerate}

\end{proof}

\end{document}
