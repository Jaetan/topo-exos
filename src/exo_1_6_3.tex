\documentclass[11pt,a4paper,twoside]{article}
\usepackage{mathtools}
\usepackage{amsfonts}
\usepackage{amssymb}
\usepackage{amsthm}
\usepackage{mathrsfs}
\usepackage[shortlabels]{enumitem}

\theoremstyle{definition}
\newcounter{excounter}
\setcounter{excounter}{2}
\newtheorem{exercise}[excounter]{Exercise}

\begin{document}

\begin{exercise}

  Let $X$ be the two-elements set $\{ 0, 1 \}$. Find a bijective correspondence between $X^\omega$
  and a proper subset of itself.

\end{exercise}

\begin{proof}

  Let $A$ be the subset of $X^\omega$ of all $\omega$-tuples of elements of $X$ that start with $0$.
  Then $A$ is a strict subset of $X^\omega$. Let
  \begin{align*}
    f : X^\omega &\to A \\
    (x_1, x_2, \dotsc) &\mapsto (0, x_1, x_2, \dotsc)
  \end{align*}
  Let $y' = (0, y_1, y_2, \dotsc) \in A$, we have $y' = f (y_1, y_2, \dotsc)$ so that $f$ is surjective.
  Now let $z = (z_1, z_2, \dotsc)$ and $y = (y_1, y_2, \dotsc)$ be elements of $X^\omega$, and suppose that $f (y) = f (z)$.
  Then $(0, y_1, y_2, \dotsc) = (0, z_1, z_2, \dotsc)$, so that $\forall i \in \mathbb{Z}_+$, $y_i = z_i$ and finally $y = z$.
  Therefore $f$ is also injective, and is thus a bijection between $X^\omega$ and its proper subset $A$.
  This shows that $X^\omega$ is infinite.

\end{proof}

\end{document}
