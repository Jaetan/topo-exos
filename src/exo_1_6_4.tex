\documentclass[11pt,a4paper,twoside]{article}
\usepackage{mathtools}
\usepackage{amsfonts}
\usepackage{amssymb}
\usepackage{amsthm}
\usepackage{mathrsfs}
\usepackage[shortlabels]{enumitem}

\theoremstyle{definition}
\newcounter{excounter}
\setcounter{excounter}{3}
\newtheorem{exercise}[excounter]{Exercise}

\begin{document}

\begin{exercise}

  Let $A$ be a nonempty finite simply ordered set.
  \begin{enumerate}[(a)]
  \item Show that $A$ has a largest element. [Hint: proceed by induction on the cardinality of $A$.]
  \item Show that $A$ has the order type of a section of the positive integers.
  \end{enumerate}

\end{exercise}

\begin{proof}\hfill

  \begin{enumerate}[(a)]

  \item Let $n$ be the cardinality of $A$. Since $A$ is nonempty, we have $n \geq 1$.
    Suppose that $n = 1$. Then $A = \{ a \}$, and since the element $a$ is the only element of $A$, it is also its largest element.
    The subset $\mathscr{A}$ of $\mathbb{Z}_+$ of elements $n$ such that any nonempty finite simply ordered set of cardinality $n$ has a largest element,
    is nonempty since $1 \in \mathscr{A}$.

    Suppose that $n \in \mathscr{A}$ and let $A$ be a nonempty finite simply ordered set of cardinality $n + 1$
    (such sets exist: for example consider the section $S_n$ of positive integers).
    Let $a_0 \in A$. The set $B = A - \{ a_0 \}$ is nonempty (since $n \geq 1$), finite (since it is a proper subset of a finite set),
    simply ordered (since elements of $B$ can be compared as elements of $A$). From this and $n \in \mathscr{A}$ we deduce that $B$ has a largest element $b_0$.
    As elements of $A$, $a_0$ and $b_0$ are comparable, and
    \begin{itemize}
    \item either $b_0 < a_0$, in which case $a_0$ is the largest element of $A$
    \item or $a_0 < b_0$, in which case $b_0$ is the largest elements of $A$
    \end{itemize}
    In both cases, we conclude to the existence of a largest element in $A$, so that $n + 1 \in \mathscr{A}$.
    As a nonempty inductive subset of $\mathbb{Z}_+$,  $\mathscr{A} = \mathbb{Z}_+$ and the result is proved.

  \item Suppose that $A$ has the same order type as a section $S_n$ of the positive integers.
    Then there exists a bijection $f : A \to S_n$ which respects order. Since $S_n$ is finite, we conclude that
    $A$ has the cardinality of $S_n$. Therefore when supposing that $A$ has the order type of a section of the positive integers,
    it is enough to consider the section of positive integers with the same cardinality as $A$.

    Suppose that $A$ has cardinality $1$, then there is a bijection $f : A \to \{ 1 \}$, and $f$ necessarily respects order.
    So $A$ has the order type of the section of the positive integers $S_0 = \{ 1 \}$.

    Suppose now that the result is true for any $A$ of cardinality $n$,
    and let $A$ be a nonempty finite simply ordered set of cardinality $n + 1$.
    From the previous point, $A$ has a largest element $a_0$. Then $B = A - \{ a_0 \}$ is a nonempty finite simply ordered set of cardinality $n$,
    so by hypothesis, it has the same order type as the section of positive integers $S_{n - 1} = \{ 1, 2, \dotsc, n \}$.
    Let $f : B \to \{ 1, 2, \dotsc, n \}$ be a bijection that respects order, and let
    \begin{align*}
      g : A &\to \{ 1, 2, \dotsc, n + 1 \} \\
      x &\mapsto \begin{cases}
        f (x) &\quad\text{ if }\quad x \in B \\
        n + 1 &\quad\text{ if }\quad x = a_0
    \end{cases}
    \end{align*}
    Then $g$ is bijective, since $f$ is bijective and $a_0 \notin B$ has a different image by $g$ than any element of $B$.
    Further we have $\forall x \in B$, $f (x) < f (a_0) = n + 1$ so $g$ respects the order of $A$, so that $A$ and $S_n$ have the same order type.

  \end{enumerate}

\end{proof}

\end{document}
