\documentclass[11pt,a4paper,twoside]{article}
\usepackage{mathtools}
\usepackage{amsfonts}
\usepackage{amssymb}
\usepackage{amsthm}
\usepackage{mathrsfs}
\usepackage[shortlabels]{enumitem}

\theoremstyle{definition}
\newcounter{excounter}
\setcounter{excounter}{5}
\newtheorem{exercise}[excounter]{Exercise}

\begin{document}

\begin{exercise}

  \begin{enumerate}[(a)]
  \item Let $A = \{ 1, 2, \dotsc, n \}$. Show there is a bijection of $\mathscr{P} (A)$
    with the cartesian product $X^n$, where $X$ is the two-elements set $X = \{ 0, 1 \}$.
  \item Show that if $A$ is finite, then $\mathscr{P} (A)$ is finite.
  \end{enumerate}

\end{exercise}

\begin{proof}\hfill

  \begin{enumerate}[(a)]

  \item Let
    \begin{align*}
      \phi : \mathscr{P} (A) &\to X^n \\
      B &\mapsto (x_1, x_2, \dotsc, x_n)
    \end{align*}
    such that
    \begin{equation*}
      \forall i \in \{ 1, 2, \dotsc n \}, \; x_i = \begin{cases}
        1 &\quad\text{ if }\quad i \in B \\
        0 &\quad\text{ otherwise }
      \end{cases}
    \end{equation*}
    Let $B1, B2 \in \mathscr{P} (A)$ such that $B_1 \neq B_2$. The roles of $B_1$ and $B_2$ being symmetric in the following,
    it is sufficient to suppose that there exists $j \in B_1 - B_2$. Then $\phi (B_1)$ and $\phi (B_2)$ have a different $j$-th
    coordinate and so $\phi (B_1) \neq \phi (B_2)$ and $\phi$ is injective.

    Let $x = (x_1, x_2, \dotsc, x_n) \in X^n$. If $x = (0, 0, \dotsc, 0)$, then $\phi (\varnothing) = x$.
    Otherwise, let $B = \{ i \in A \mid x_i = 1 \}$. We have $\phi (B) = (y_1, y_2, \dotsc, y_n)$, with
    $y_i = 1$ if and only if $i \in B$, which, in turns, is true if and only if $x_i = 1$, so that $\phi (B) = x$
    and $\phi$ is surjective.

  \item Let $A$ be a finite set. If $A = \varnothing$, then $\mathscr{P} (A) = \varnothing$, so $\mathscr{P} (A)$ is finite.
    Otherwise, there is a bijection $f$ from $\mathscr{P} (A)$ to $X^n$ for some positive integer $n$.
    Since $X$ is finite, the cartesian product $X^n$ is also finite, so there is a bijection $g$ from $X^n$ to some section
    of positive integers $\{ 1, 2, \dotsc, m \}$. Therefore $g \circ f$ is a bijection from $\mathscr{P} (A)$ to $\{ 1, 2, \dotsc, m \}$,
    so $\mathscr{P} (A)$ is finite.

  \end{enumerate}

\end{proof}

\end{document}
