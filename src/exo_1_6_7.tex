\documentclass[11pt,a4paper,twoside]{article}
\usepackage{mathtools}
\usepackage{amsfonts}
\usepackage{amssymb}
\usepackage{amsthm}
\usepackage{mathrsfs}
\usepackage[shortlabels]{enumitem}

\theoremstyle{definition}
\newcounter{excounter}
\setcounter{excounter}{6}
\newtheorem{exercise}[excounter]{Exercise}

\begin{document}

\begin{exercise}

  If $A$ and $B$ are finite, show that the set of all functions $f : A \to B$ is finite.

\end{exercise}

\begin{proof}

  Let $D$ be the set of all functions from $A$ to $B$.
  If $B = \varnothing$, then there is no function from $A$ to $B$, so $D$ is empty, and thus finite.
  If $A = \varnothing$, then there is one function $f$ from $A$ to $B$ (the function whose rule of assignment is $\varnothing$),
  so $D = \{ f \}$ and is therefore finite.

  Otherwise, let
  \begin{align*}
    \phi : D &\to \mathscr{P} (A \times B) \\
    f &\mapsto \big\{ \big(a, f (a) \big) \mid a \in A \big\}
  \end{align*}
  Let $f, g : A \to B$ be distinct functions. There exists $x \in A$ such that $f (x) \neq g (x)$, from which we deduce that
  $\phi (f) \neq \phi (g)$, so that $\phi$ is injective.
  Since $A$ and $B$ are finite, the cartesian product $A \times B$ is finite, and so is $\mathscr{P} (A \times B)$;
  let $h : \mathscr{P} (A \times B) \to \{ 1, 2, \dotsc, n \}$
  be a bijection, for some $n$. Then $h \circ \phi$ is an injection from $D$ to $\{ 1, 2, \dotsc, n \}$, so that $D$ is finite.

\end{proof}

\end{document}
