\documentclass[11pt,a4paper,twoside]{article}
\usepackage{mathtools}
\usepackage{amsfonts}
\usepackage{amssymb}
\usepackage{amsthm}
\usepackage{mathrsfs}
\usepackage[shortlabels]{enumitem}

\theoremstyle{definition}
\newcounter{excounter}
\setcounter{excounter}{2}
\newtheorem{exercise}[excounter]{Exercise}

\begin{document}

\begin{exercise}

  Let $X$ be the two-element set $\{ 0, 1 \}$. Show there is a bijective correspondence
  between the set $\mathscr{P} ( \mathbb{Z}_+ )$ and the cartesian product $X^\omega$.

\end{exercise}

\begin{proof}

  Let
  \begin{align*}
    f : \mathscr{P} ( \mathbb{Z}_+ ) &\to X^\omega \\
    A &\mapsto (x_1, x_2, \dotsc) &\text{ with } &&x_i = \begin{cases}
      1 &\text{ if } i \in A \\
      0 &\text{ otherwise }
    \end{cases}
  \end{align*}
  Let $(x_1, x_2, \dotsc) \in X^\omega$. The expression $S = \{ i \in \mathbb{Z}_+ \mid x_i = 1 \}$ defines a subset of $\mathbb{Z}_+$.
  If $S \neq \varnothing$, then we have $f (S) = (x_1, x_2, \dotsc)$. Otherwise, $x_i = 0$ for all $i$, and $f (\varnothing) = (0, 0, \dotsc)$.
  From this we deduce that $f$ is surjective.

  Let now $A, B$ be distinct subsets of $\mathbb{Z}_+$. By switching the roles of $A$ and $B$ if needed, we can suppose that $A \neq \varnothing$.
  Since $A$ and $B$ are distinct, there exists $j \in A - B$. Then for $f (A) = (a_1, a_2, \dotsc, a_j, \dotsc)$ and $f (B) = (b_1, b_2, \dotsc, b_j, \dotsc)$,
  we have $a_j = 1$ but $b_j = 0$, so $f (A) \neq f (B)$ and $f$ is injective.

\end{proof}

\end{document}
