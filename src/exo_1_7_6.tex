\documentclass[11pt,a4paper,twoside]{article}
\usepackage{mathtools}
\usepackage{amsfonts}
\usepackage{amssymb}
\usepackage{amsthm}
\usepackage{mathrsfs}
\usepackage[shortlabels]{enumitem}

\theoremstyle{definition}
\newcounter{excounter}
\setcounter{excounter}{5}
\newtheorem{exercise}[excounter]{Exercise}

\begin{document}

\begin{exercise}

  We say that two sets $A$ and $B$ \emph{have the same cardinality} if there is a bijection of $A$ with $B$.

  \begin{enumerate}[(a)]

  \item Show that if $B \subset A$ and there is an injection
    \begin{equation*}
      f : A \to B
    \end{equation*}
    the $A$ and $B$ have the same cardinality. [\emph{Hint:} Define $A_1 = A$, $B_1 = B$,
      and for $n > 1$, $A_n = f (A_{n - 1})$ and $B_n = f (B_{n - 1})$. (Recursive definition again!)
      Note that $A_1 \supset B_1 \supset A_2 \supset B_2 \supset A_3 \supset \dotsb$. Define a bijection
      $h : A \to B$ by the rule
      \begin{equation*}
        h (x) = \begin{cases}
          f (x) &\text{ if } x \in A_n - B_n \text{ for some } n \\
          x &\text{ otherwise.] }
        \end{cases}
      \end{equation*}

    \item \emph{Theorem (Schr\"oder-Bernstein theorem). \quad If there are injections $f : A \to C$ and $g : C \to A$,
    then $A$ and $C$ have the same cardinality.}

  \end{enumerate}

\end{exercise}

\begin{proof}\hfill

  \begin{enumerate}[(a)]

  \item \label{it:a}
    Let $A_i, B_i$ be defined as above for all $i \in \mathbb{Z}_+$, and let $\mathscr{A}$ be the subset of $\mathbb{Z}_+$
    such that for all $n \in \mathscr{A}$, $A_n \supset B_n \supset A_{n + 1}$. Then $1 \in \mathscr{A}$: we already have
    $B_1 = B \subset A = A_1$; from the definition of $f$, we also have $A_2 = f (A_1) = f (A) \subset B = B_1$.

    Suppose now that $S_{n + 1} \subset \mathscr{A}$ for some $n$. Since $B_n \subset A_n$, we have $B_{n + 1} = f (B_n) \subset f (A_n) = A_{n + 1}$.
    Similarly, $A_{n + 1} \subset B_n$, so $A_{n + 2} = f (A_{n + 1}) \subset f (B_n) = B_{n + 1}$. From this we deduce that $S_{n + 2} \subset \mathscr{A}$,
    so that $\mathscr{A}$ is inductive, and finally $\mathscr{A} = \mathbb{Z}_+$.

    Since $f : A \to B$ exists, we have $B \neq \varnothing$. And since $B \subset A$, we also have $A \neq \varnothing$.
    If $A$ has only one element, then $B = A$, and $h = i_A$, which is bijective.

    \begin{description}

    \item[Injectivity of $h$] \hspace{0pt}\\
      Suppose then that there are $x, y \in A$ such that $x \neq y$. If we have $h (x) = x$ and $h (y) = y$, then $h (x) \neq h (y)$.
      Similarly, if we have $h (x) = f (x)$ and $h (y) = f (y)$, then $h (x) \neq h (y)$ since $f$ is injective.

      Otherwise, we can suppose that $h (x) = f (x)$ and $h (y) = y$, possibly switching the roles of $x$ and $y$.
      Then there exists $n$ such that $x \in A_n - B_n$, but for all $n \in \mathbb{Z}_+$, $y \notin A_n - B_n$.
      Therefore $y \in \cup_{n \in \mathbb{Z}_+} B_n - A_{n + 1}$, but $f (x) \in f (A_n) = A_{n + 1}$, so that $h (x) \neq h (y)$.
      From the above, we deduce that $h$ is injective.

    \item[Surjectivity of $h$] \hspace{0pt}\\
      Let $y \in B$. If $y \in \mathscr{B} = \cup_{n \in \mathbb{Z}_+} B_n - A_{n + 1}$, then $h (y) = y$.

      Otherwise, there exists $n$ such that $y \in A_n - B_n$. Since we also have $y \in B$, then $n \geq 2$ and $y \in f (A)$.
      From this we deduce that $f^{-1} ( \{ y \} ) \neq \varnothing$.

      If $f^{-1} ( \{ y \} ) \cap \mathscr{B} \neq \varnothing$, then take $x$ in it.
      There exists $m \in \mathbb{Z}_+$ such that $x \in B_m - A_{m + 1}$, so that $f (x) = y \in f (B_m - A_{m + 1})$.
      Since $f$ is injective, we have $f (B_m - A_{m + 1}) = f (B_m) - f (A_{m + 1}) = B_{m + 1} - A_{m + 2}$.

      From $y \in A_n$ and $y \in B_{m + 1}$, we deduce that $m + 1 < n$ and $m + 2 \leq n$. We must therefore have
      $y \notin A_{m + 2} \supset A_n$. This contradicts $y \in A_n$, so $f^{-1} ( \{ y \} ) \cap \mathscr{B}$ must be empty.

      So there exists $p \in \mathbb{Z}_+$ such that $x \in A_p - B_p$; in this case, we have $h (x) = f (x) = y$.

    \end{description}

    From the above we deduce that $h$ is bijective, so that $A$ and $B$ have the same cardinality.

  \item Let $f : A \to C$ and $g : C \to A$ be injections and $B = g (C) \subset A$.
    The function $g \circ f : A \to B$ is injective, so we are in the hypotheses of \ref{it:a}.
    There exists a bijection $h : A \to g (C)$. Furthermore, $g$ is surjective from $C$ to $g (C)$,
    so $k : C \to B$, $x \mapsto g (x)$ is bijective. Therefore the function $h^{-1} \circ k : C \to A$
    is a bijection, and $A$ and $C$ have the same cardinality.

  \end{enumerate}

\end{proof}

\end{document}
