\documentclass[11pt,a4paper,twoside]{article}
\usepackage{mathtools}
\usepackage{amsfonts}
\usepackage{amssymb}
\usepackage{amsthm}
\usepackage{mathrsfs}
\usepackage[shortlabels]{enumitem}

\theoremstyle{definition}
\newcounter{excounter}
\setcounter{excounter}{3}
\newtheorem{exercise}[excounter]{Exercise}

\begin{document}

\begin{exercise}

  The \emph{Fibonacci numbers} of number theory are defined recursively by the formula
  \begin{align*}
    \lambda_1 &= \lambda_2 = 1 \\
    \lambda_n &= \lambda_{n - 1} + \lambda_{n - 2} &\text{ for } n > 2.
  \end{align*}
  Define them rigorously by use of Theorem 8.4.

\end{exercise}

\begin{proof}

  Let $A = \mathbb{Z}_+$ and $\mathscr{A}$ be the set of functions mapping a nonempty section of the positive integers into $\mathbb{Z}_+$.
  Define
  \begin{align*}
    \rho : \mathscr{A} &\to \mathbb{Z}_+ \\
    f &\mapsto \begin{cases}
      1 &\text{ if the domain of f is } \{ 1 \} \\
      f (n) + f (n - 1) &\text{ otherwise }
    \end{cases}
  \end{align*}
  In the last case, the domain of $f$ is a section $\{ 1, \dotsc, n \}$ of the positive integers with $n > 1$. Let $a_0 = 1$, and apply
  Theorem 8.4 with these values of $a_0$ and $\rho$ to deduce the existence of a function $h$ such that:
  \begin{align*}
    h (1) &= a_0 = 1 \\
    h (i) &= \rho (h | \{ 1, \dotsc, i - 1 \}) &\text{ if } i > 1
  \end{align*}
  For $i = 2$, we get
  \begin{equation*}
    h (2) = \rho ( h | \{ 1 \} ) = 1
  \end{equation*}
  and for $i > 2$, we get
  \begin{equation*}
    h (i) = \rho ( h | \{ 1, \dotsc, i - 1 \} ) = h (i - 1) + h (i - 2)
  \end{equation*}
  so that $h (i)$ is the $i$-th Fibonacci number.

\end{proof}

\end{document}
