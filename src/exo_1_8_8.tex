\documentclass[11pt,a4paper,twoside]{article}
\usepackage{mathtools}
\usepackage{amsfonts}
\usepackage{amssymb}
\usepackage{amsthm}
\usepackage{mathrsfs}
\usepackage[shortlabels]{enumitem}

\theoremstyle{definition}
\newcounter{excounter}
\setcounter{excounter}{7}
\newtheorem{exercise}[excounter]{Exercise}

\begin{document}

\begin{exercise}

  Verify the following version of the principle of recursive definition: let $A$ be a set.
  Let $\rho$ be a function assigning, to every function $f$ mapping a section $S_n$ of $\mathbb{Z}_+$
  into $A$, an element $\rho (f)$ of $A$. Then there exists a unique function $h : \mathbb{Z}_+ \to A$
  such that $h (n) = \rho ( h | S_n )$ for each $n \in \mathbb{Z}_+$.

\end{exercise}

\begin{proof}

  Let $\rho'$ be a function assigning, to every function $f$ mapping a nonempty section $S_{n + 1}$ of $\mathbb{Z}_+$
  into $A$, an element $\rho' (f) = \rho (f)$, and let $a_0 = \rho ( \varnothing ) \in A$. Theorem 8.4 allows us to
  conclude that there exists a unique function $h : \mathbb{Z}_+ \to A$ such that:
  \begin{align*}
    h (1) &= a_0 = \rho ( \varnothing ) \\
    h (n) &= \rho' ( h | S_n ) = \rho ( h | S_n ) &\text{ for } n > 1
  \end{align*}
  which is the expected result.

\end{proof}

\end{document}
