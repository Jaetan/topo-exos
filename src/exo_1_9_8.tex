\documentclass[11pt,a4paper,twoside]{article}
\usepackage{mathtools}
\usepackage{amsfonts}
\usepackage{amssymb}
\usepackage{amsthm}
\usepackage{mathrsfs}
\usepackage[shortlabels]{enumitem}

\theoremstyle{definition}
\newcounter{excounter}
\setcounter{excounter}{7}
\newtheorem{exercise}[excounter]{Exercise}

\begin{document}

\begin{exercise}

  Show that $\mathscr{P} ( \mathbb{Z}_+ )$ and $\mathbb{R}$ have the same cardinality.
  [ \emph{Hint:} you may usr the fact that every real number has a decimal expansion, which is unique if
  expansions that end in an infinite string of $9$s are forbidden. ]

\end{exercise}

\begin{proof}\hfill

  \begin{description}

    \item[Injection of $\mathbb{R}$ into $\mathscr{P} ( \mathbb{Z}_+ )$] \hspace{0pt}\\
      For all $x \in [ 0, 1 ]$, we admit that $x$ has a unique proper base $2$ development noted $\left( 0 , x_1 x_2 \dotsc \right)_2$,
      with $x_i \in \{ 0, 1 \}$ for all $i \in \mathbb{Z}_+$, and such that for all $i \in \mathbb{Z}_+$ there exists $j > i$ such that $x_j = 0$.
      Let
      \begin{align*}
        \phi : ( 0, 1 ) &\to \mathscr{P} ( \mathbb{Z}_+ ) \\
        x = \left( 0 , x_1, x_2 \dotsc \right)_2 &\mapsto \{ i \in \mathbb{Z}_+ \mid x_i = 1 \}
      \end{align*}
      The unicity of the proper base $2$ development of $x$ implies that if $x, y$ are elements of $( 0, 1 )$ such that $x \neq y$, then there exists $j \in \mathbb{Z}_+$
      such that $x_j \neq y_j$; so that $j \notin \phi ( x ) \cap \phi ( y )$, from which we deduce that $\phi$ is injective.

      Note that $\phi$ is not surjective: suppose that there exists $x = \left( 0, x_1, x_2 \dotsc \right)_2$ such that $\phi ( x ) = \mathbb{Z}_+$.
      If $x_i = 0$ for some $i \in \mathbb{Z}_+$, then $i \notin \phi ( x ) = \mathbb{Z}_+$, which is a contradiction. Therefore $x_i = 1$ for all $i$,
      so that $\left( 0 , x_1 x_2 \dotsc \right)_2$ is not a proper base $2$ development.

      Let
      \begin{align*} \label{def:bij_R_01}
        \theta : \mathbb{R} &\to ( 0, 1 ) \\
        x &\mapsto \frac{1}{2} + \frac{1}{\pi} \arctan \left( x \right)
      \end{align*}
      $\theta$ is a bijection, and thus $\phi \circ \theta$ is an injection of $\mathbb{R}$ into $\mathscr{P} ( \mathbb{Z}_+ )$.

    \item[Injection of $\mathscr{P} ( \mathbb{Z}_+ )$ into $\mathbb{R}$] \hspace{0pt}\\
      Let $X$ be a nonempty subset of $\mathbb{Z}_+$. For all $n \in \mathbb{Z}_+$, let
      \begin{equation*}
        S_n ( X ) = \sum_{i \in X \cap \{ 1, \dotsc, n \}} 2^{- i}
      \end{equation*}
      We have
      \begin{equation*}
        0 \leq S_n ( X ) \leq S_{n + 1} ( X ) \leq \sum_{i \in \{ 1, \dotsc, n + 1 \}} 2^{- i} \leq 1
      \end{equation*}
      so that the sequence $S_n ( X )$ converges to some real number $S ( X )$ in $[ 0, 1 ]$.
      For all $n \in \mathbb{Z}_+$, let $Q_n = \{ i \in \mathbb{Z}_+ \mid i \geq n \}$, and
      \begin{align*}
        \psi : \mathscr{P} ( \mathbb{Z}_+ ) &\to \mathbb{R} \\
        X &\mapsto \begin{cases}
          0 &\text{ if } X = \varnothing \\
          S ( X ) &\text{ if } Q_n \not\subset X \text{ for all } n \\
          1 + S ( X ) &\text{ otherwise }
        \end{cases}
      \end{align*}
      Let $X, Y$ be subsets of $\mathbb{Z}_+$ such that $X \neq Y$. Switching the roles of $X$ and $Y$ if necessary,
      we can suppose that $X - Y \neq \varnothing$. Let $i \in X - Y$; if $Y = \varnothing$, then $\psi ( Y ) = 0 < 2^{- i} \leq \psi ( X )$,
      so $\psi ( X ) \neq \psi ( Y )$. Otherwise, if there is some $n$ such that $Q_n \subset Y$, but for all $n$, $Q_n \not\subset X$, then
      $\psi ( X ) \leq 1 < 1 + \sum_{i \in Q_n} 2^{- i} \leq \psi ( Y )$, so that $\psi ( X ) \neq \psi ( Y )$.
      The same reasoning applies if $X$ contains some $Q_n$ but $Y$ does not contain any.

      Otherwise, suppose that neither $X$ nor $Y$ contains any $Q_n$, and let $m = \min ( X - Y )$.
      Switching the roles of $X$ and $Y$ if necessary, we can suppose that $x_m = 1$. For all $i < m$, we have $x_i = y_i$, so that
      \begin{align*}
        0 \leq \psi ( Y ) &< \sum_{i \in \{ 1, \dotsc, m - 1 \} \cap Y} 2^{- i} + \sum_{i > m} 2^{- i} &\text{ since } Q_{m + 1} \not\subset Y \\
        &= \sum_{i \in \{ 1, \dotsc, m - 1 \} \cap X} 2^{-i} + 2^{- m} &\text{ since } x_i = y_i \text{ for } i < m \\
        &\leq \psi ( X )
      \end{align*}
      From this we deduce that $\psi ( X ) \neq \psi ( Y )$.

      The last case to check is when there is some positive $n$ such that $Q_n \subset X$ and some $p$ such that $Q_p \subset Y$.
      Again let $m = \min (X - Y)$, and suppose that $x_m = 1$, switching the roles of $X$ and $Y$ if necessary. Then
      \begin{align*}
        \psi ( Y ) &\leq 1 + \sum_{i \in \{ 1, \dotsc, m - 1 \} \cap Y} 2^{- i} + \sum_{i > m} 2^{- i} \\
        &< 1 + \sum_{i \in \{ 1, \dotsc, m - 1 \} \cap X} 2^{- i} + 2^{- m} \\
        &\leq \psi ( X )
      \end{align*}
      and we have $\psi ( X ) \neq \psi ( Y )$.

      From the above, we deduce that $\psi$ is injective.

  \end{description}

  Since there exists an injection of $\mathbb{R}$ into $\mathscr{P} ( \mathbb{Z}_+ )$, and an injection of $\mathscr{P} ( \mathbb{Z}_+ )$ into $\mathbb{R}$,
  there exists a bijection from $\mathscr{P} ( \mathbb{Z}_+ )$ to $\mathbb{R}$, so that they have the same cardinality.

\end{proof}

\end{document}
