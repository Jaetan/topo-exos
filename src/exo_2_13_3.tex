\documentclass[11pt,a4paper,twoside]{article}
\usepackage{mathtools}
\usepackage{amsfonts}
\usepackage{amssymb}
\usepackage{amsthm}
\usepackage{mathrsfs}
\usepackage{cleveref}
\usepackage[shortlabels]{enumitem}
\usepackage{parskip}

\theoremstyle{definition}
\newcounter{excounter}
\setcounter{excounter}{3}
\newtheorem{exercise}[excounter]{Exercise}

\newcounter{examplecounter}
\setcounter{examplecounter}{3}
\theoremstyle{plain}
\newtheorem{example}[examplecounter]{Example}

\begin{document}

\begin{exercise}

  Show that the collection $\mathscr{T}_c$ given in example 4 of \S 12 is a topology on the set $X$.
  Is the collection
  \begin{equation*}
    \mathscr{T}_\infty = \{ U \mid X - U \text{ is infinite or empty or all of } X \}
  \end{equation*}
  a topology on $X$?

\end{exercise}

\bigskip
\begin{example}
  Let $X$ be a set; let $\mathscr{T}_c$ be the collection of all subsets $U$ of $X$ such that $X - U$
  is either countable or is all of $X$. Then $\mathscr{T}_c$ is a topology on $X$.
\end{example}
  
\begin{proof}

  Let us show that $\mathscr{T}_c$ satisfies the definition of a topology.

  \begin{enumerate}[(a)]

  \item $X - X = \varnothing$, which is countable, so $X \in \mathscr{T}_c$.

  \item $X - \varnothing = X$, which is all of $X$, so $\varnothing \in \mathscr{T}_c$.

  \item For all nonempty family $\{ U_\alpha \}_{ \alpha \in J }$ of elements of $X$, we have
    \begin{equation*}
      X - \bigcup_{ \alpha \in J } \, U_\alpha = \bigcap_{ \alpha \in J } \, ( X - U_\alpha )
    \end{equation*}
    and the latter is countable as a subset of $X - U_\alpha$ for all $\alpha \in J$. Thus the
    union $\cup_{\alpha \in J} \, U_\alpha$ is also an element of $X$.

  \item For all finite set $\{ U_1, \dotsc , U_n \}$ of elements of $X$, we have
    \begin{equation*}
      \bigcap_{k = 1}^n \, ( X - U_k ) = X - \bigcup_{k = 1}^n \, U_k
    \end{equation*}
    From the previous point, the union $\cup_{k = 1}^n \, U_k$ of elements of $X$ is
    an element $U$ of $X$, and so $X - U$ is either countable or equal to $X$. Thus
    the finite intersection $\cap_{k = 1}^n \, U_k$ is an element of $X$.

  \end{enumerate}

  The family $\mathscr{T}_\infty = \{ U \in X \mid X - U \text{ is infinite or empty or all of } X \}$ is
  not in general a topology on $X$: let $X = \mathbb{R}$, the family
  \begin{equation*}
    F = \{ ( {- \infty}, - 1 / n ) \cup ( 1 / n, {+ \infty} ), n \in \mathbb{Z}_+ \}
  \end{equation*}
  satisfies $X - U$ is infinite for all $U \in F$ and is therefore a subset of $\mathscr{T}_\infty$.
  However,
  \begin{equation*}
    \bigcap_{U \in F} \, ( X - U ) = \bigcap_{k = 1}^\infty \, [ - \frac{1}{k}, \frac{1}{k} ] = \{ 0 \}
  \end{equation*}
  which is not infinite, not empty, and not equal to $\mathbb{R}$, so $\cup_{U \in F} \, U$ is not an
  element of $\mathscr{T}_\infty$.

\end{proof}

\end{document}
