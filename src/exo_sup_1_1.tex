\documentclass[11pt,a4paper,twoside]{article}
\usepackage{mathtools}
\usepackage{amsfonts}
\usepackage{amssymb}
\usepackage{amsthm}
\usepackage{mathrsfs}
\usepackage{cleveref}
\usepackage[shortlabels]{enumitem}
\usepackage{parskip}

\theoremstyle{definition}
\newcounter{excounter}
\setcounter{excounter}{0}
\newtheorem{exercise}[excounter]{Exercise}

\theoremstyle{plain}
\newtheorem*{theorem}{Theorem}
\newtheorem{lemma}{Lemma}

\begin{document}

\begin{exercise}

  \begin{theorem}[General principle of recursive definition]
    Let $J$ be a well-ordered set; let $C$ be a set.
    Let $\mathscr{F}$ be the set of all functions mapping sections of $J$ into $C$. Given a function
    $\rho : \mathscr{F} \to C$, there exists a unique function $h : J \to C$ such that
    $h ( \alpha ) = \rho ( h | S_\alpha )$ for each $\alpha \in J$.

  \end{theorem}
  [\emph{Hint:} Follow the pattern outlined in exercise 10 of \S 10.]

\end{exercise}

\begin{proof}

  For future reference in the proof, the property we wish to have a function $f : J \to C$ satisfy is
  \begin{equation} \label{eq:induction}
    \forall \alpha \in J, \quad h ( \alpha ) = \rho ( f | S_\alpha )
  \end{equation}

  The existence of a function $f : J \to C$ implies that $C$ is nonempty.
  Furthermore, if $J$ is empty, then there is a unique function from $J$ to $C$, and it vacuously
  verifies \eqref{eq:induction}. Therefore we assume in the rest of the proof that $J$ is nonempty.

\begin{lemma} \label{lemma:unicity}

  If $h$ and $k$ map sections of $J$ into $C$ and satisfy \eqref{eq:induction}
  for all $\alpha$ in their respective domains, then that $h ( x ) = k ( x )$ for all $x$ in both domains.

\end{lemma}

\begin{proof}

  Let $\mathscr{D}_h$ (resp. $\mathscr{D}_k$) be the domain of $h$ (resp. of $k$). The set $\mathscr{D}_h \cap \mathscr{D}_k$,
  as the intersection of two sections of $J$, is equal to one of these sections. Possibly switching the roles
  of $h$ and $k$, we can suppose that $\mathscr{D}_h \subset \mathscr{D}_k$.
  Let $\mathscr{A}$ be the subset of $\mathscr{D}_h$ such that for all $\alpha \in \mathscr{A}$, $h ( \alpha ) \neq k ( \alpha )$.
  If $\mathscr{A}$ is nonempty, then, as a subset of the well-ordered set $J$, it has a smallest element $\beta$.
  For all $\alpha < \beta$, $\alpha$ is an element of the section $\mathscr{D}_h$ and $h ( \alpha ) = k ( \alpha )$,
  so that $h | S_\beta = k | S_\beta$. Since $h$ and $k$ satisfy \eqref{eq:induction}, we deduce that
  $h ( \beta ) = \rho ( h | S_\beta ) = \rho ( k | S_\beta ) = k ( \beta )$, which contradicts the existence of $\beta$.
  Therefore for all $\alpha \in \mathscr{D}_h$, we have $h ( \alpha ) = k ( \alpha )$.

\end{proof}

\begin{lemma} \label{lemma:inductive_step}

  If there exists a function $h : S_\alpha \to C$ which satisfies \eqref{eq:induction}, then there exists a function
  $k : S_\alpha \cup \{ \alpha \} \to C$ which satisfies \eqref{eq:induction}.

\end{lemma}

\begin{proof}

  Let $\alpha \in J$ and suppose there exists $h : S_\alpha \to C$ satisfying \eqref{eq:induction}. Let
  \begin{align*} \label{eq:induction}
    k : S_\alpha \cup \{ \alpha \} &\to C \\
    x &\mapsto \begin{cases}
      h ( x ) &\text{if } x \in S_\alpha \\
      \rho ( h ) &\text{otherwise}
    \end{cases}
  \end{align*}
  For all $x \in S_\alpha$, $k ( x ) = h ( x )$, so that for all $y \in S_x$, $h | S_x = k | S_x$. Since $h$ satisfies \eqref{eq:induction},
  we have $k ( x ) = h ( x ) = \rho ( h | S_x ) = \rho ( k | S_x )$. Thus $k$ satisfies \eqref{eq:induction} on $S_\alpha$.
  Moreover, $k ( \alpha ) = \rho ( h ) = \rho ( k | S_\alpha )$, so $k$ satisfies \eqref{eq:induction} on $S_\alpha \cup \{ \alpha \}$.

\end{proof}

\begin{lemma} \label{lemma:limit_case}

  If $K \subset J$ and for all $\alpha \in K$, there exists a function $h_\alpha : S_\alpha \to C$ satisfying \eqref{eq:induction},
  there exists a function
  \begin{align*}
    k : \bigcup_{\alpha \in K} \,S_\alpha &\to C
  \end{align*}
  satisfying \eqref{eq:induction}.

\end{lemma}

\begin{proof}

  Let $I = \cup_{\alpha \in K} \,S_\alpha$. If $K$ is empty or has only one element, then $I$ is empty and there is no function
  from $I$ to $C$. Therefore $K$ has at least two elements.

  Let $R$ be the subset of $I \times C$ containing all the tuples $( x, h_\alpha ( x ) )$ for all $\alpha \in K$ and all $x \in S_\alpha$.
  Let $\alpha, \beta \in K$ such that $\alpha < \beta$; we have $S_\alpha \subset S_\beta$, and from \cref{lemma:unicity},
  $h_\beta | S_\alpha = h_\alpha$. From this we deduce that for all $x \in I$, there is a unique $y \in C$ such that $( x, y ) \in R$.
  Therefore $R$ is the rule of a function $k : I \to C$.

  Let $x \in I$, there exists $\alpha \in K$ such that $x \in S_\alpha$. The function $h_\alpha$ satisfies \eqref{eq:induction} on its domain $S_\alpha$,
  so $k ( x ) = h_\alpha ( x ) = \rho ( h_\alpha | S_x )$. For all $y \in S_x$, $h_\alpha ( y )$ is the unique element of $R$ with $y$ as its
  first coordinate, so $h_\alpha ( y ) = k ( y )$. From this we deduce that $h_\alpha | S_x = k | S_x$, and therefore $k$ satisfies \eqref{eq:induction}.

\end{proof}

\begin{lemma} \label{lemma:transfinite_induction}

  For all $\beta \in J$, there exists a function $h_\beta : S_\beta \to C$ satisfying \eqref{eq:induction}.

\end{lemma}

\begin{proof}

  Let $J_0$ be the subset of $J$ such that for all $\alpha \in J_0$, there exists a function $h_\alpha : S_\alpha \to C$ satisfying \eqref{eq:induction}.
  Since $J$ is nonempty and well-ordered, it has a smallest element $m$, and $S_m = \varnothing$.
  There is only one function from $\varnothing$ to $C$, and it vacuously satisfies \eqref{eq:induction},
  so $S_m \subset J_0$, $m \in J_0$ and $J_0$ is nonempty.

  Suppose that for some $\beta \in J$, we have $S_\beta \subset J_0$.
  \begin{itemize}

  \item If $\beta$ has an immediate predecessor $\alpha$, then $\alpha \in S_\beta$, so that $\alpha \in J_0$.
    Therefore there exists a function $h_\alpha : S_\alpha \to C$ satisfying \eqref{eq:induction}. The fact that
    $\alpha$ is the immediate predecessor of $\beta$ implies that $S_\beta = S_\alpha \cup \{ \alpha \}$.
    From \cref{lemma:inductive_step} we deduce that there exists a function $h_\beta : S_\beta \to C$ satisfying \eqref{eq:induction}.
    Therefore $\beta \in J_0$.

  \item Otherwise, $\beta$ does not have an immediate predecessor, and thus $S_\beta = \cup_{\alpha < \beta} \,S_\alpha$.
    For all $\alpha \in J$ such that $\alpha < \beta$, we have $\alpha \in S_\beta$ and therefore $\alpha \in J_0$.
    From this we deduce that there exists a function $h_\alpha : S_\alpha \to C$ satisfying \eqref{eq:induction}.
    We are within the hypotheses of \cref{lemma:limit_case}, and can conclude to the existence of a function $h_\beta : S_\beta \to C$
    satsifying \eqref{eq:induction}. From this we deduce that $\beta \in J_0$.

  \end{itemize}
  The above shows that $J_0$ is an inductive subset of the well-ordered set $J$, and therefore $J_0 = J$.

\end{proof}

If $J$ has a largest element $M$, then $J = S_M \cup \{ M \}$. From \cref{lemma:transfinite_induction}, we deduce the existence
of $h_M : S_M \to C$ satisfying \eqref{eq:induction}; and from \cref{lemma:inductive_step}, we deduce the existence of
a function $h : J \to C$ satisfying \eqref{eq:induction}.

Otherwise, $J = \cup_{\alpha \in J} \,S_\alpha$: for all $\alpha \in J$, $S_\alpha \subset J$, so that $\cup_{\alpha \in J} \,S_\alpha \subset J$.
Conversely, for all $\alpha \in J$, there exists $\beta \in J$ such that $\alpha < \beta$, since $J$ does not have a largest element.
From this we deduce that $\alpha \in S_\beta$ and finally $J \subset \cup_{\alpha \in J} \,S_\alpha$.

For all $\alpha \in J$, we deduce from \cref{lemma:transfinite_induction} the existence of a function
$h_\alpha : S_\alpha \to C$ satisfying \eqref{eq:induction}. Let then $R$ be the subset of the cartesian product $J \times C$
containing all the tuples $( x, h_\alpha (x) )$ for all $\alpha \in J$ and all $x \in S_x$. From \cref{lemma:unicity},
we deduce that, for all $\alpha, \beta \in J$ such that $\alpha < \beta$, $h_\beta | S_\alpha = h_\alpha$.
The set $R$ is therefore the rule of a function $h : J \to C$.

Let $\alpha \in J$. For all $x \in S_\alpha$, $( x, h_\alpha (x) )$ is the unique tuple in $R$ with $x$ as its first coordinate,
so that $h_\alpha | S_x = h | S_x$. Furthermore, since $h_\alpha$ satisfies \eqref{eq:induction} on $S_\alpha$, we have
\begin{equation*}
  h ( x ) = h_\alpha ( x ) = \rho ( h_\alpha | S_x ) = \rho ( h | S_x )
\end{equation*}
so that the function $h : J \to C$ satisfies \eqref{eq:induction}.

Suppose that $h, k : J \to C$ both satisfy \eqref{eq:induction}, and let $\mathscr{A}$ be the subset of $J$ such that
for all $\alpha \in \mathscr{A}$, $h ( \alpha ) \neq k ( \alpha )$. If $\mathscr{A}$ is nonempty, then it has a smallest element $\beta$.
The functions $h$ and $k$ are both defined on the section $S_\beta$, and for all $\alpha \in S_\beta$, $h ( \alpha ) = k ( \alpha )$.
From this we deduce that $h | S_\beta = k | S_\beta$, and, since $h$ and $k$ satisfy \eqref{eq:induction} on $S_\beta$, we have
$h ( \beta ) = \rho ( h | S_\beta ) = \rho ( k | S_\beta ) = k ( \beta )$, which contradicts the existence of $\beta$.
Therefore the function $h$ defined in the preceding paragraph is unique.

\end{proof}

\end{document}
