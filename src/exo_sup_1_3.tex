\documentclass[11pt,a4paper,twoside]{article}
\usepackage{mathtools}
\usepackage{amsfonts}
\usepackage{amssymb}
\usepackage{amsthm}
\usepackage{mathrsfs}
\usepackage{cleveref}
\usepackage[shortlabels]{enumitem}
\usepackage{parskip}

\theoremstyle{definition}
\newcounter{excounter}
\setcounter{excounter}{2}
\newtheorem{exercise}[excounter]{Exercise}

\begin{document}

\begin{exercise}

  Let $J$ and $E$ be well-ordered sets; suppose there is an order-preserving map $k : J \to E$.
  Using Exercises 1 and 2, show that $J$ has the order type of $E$ or a section of $E$. [\emph{Hint:}
    choose $e_0 \in E$. Define $h : J \to E$ by the recursion formula
    \begin{equation*}
      h ( \alpha ) = \text{smallest } \big( E - h ( S_\alpha ) \big) \quad\text{if } h ( S_\alpha ) \neq E,
    \end{equation*}
    and $h ( \alpha ) = e_0$ otherwise. Show that $h ( \alpha ) \leq k ( \alpha )$ for all $\alpha$;
  conclude that $h ( S_\alpha ) \neq E$ for all $\alpha$.]

\end{exercise}

\begin{proof}

  The existence of $k$ implies that $E$ is nonempty. If $J$ is empty, then there is no bijection between $J$ and $E$
  or a section of $E$. Therefore we suppose that $J$ is nonempty. Suppose that $k : J \to E$ is an order-preserving map,
  and let $e_0 \in E$.

  Let $\mathscr{F}$ be the set of all functions mapping sections of $J$ into $E$. For all $f \in \mathscr{F}$, let $S_\alpha$
  for some $\alpha$ in $J$, be the domain of $f$. Let
  \begin{align*}
    \rho : \mathscr{F} &\to E \\
    f &\mapsto \begin{cases}
      \min \big( E - f ( S_\alpha )  \big) &\text{if } f ( S_\alpha ) \neq E \\
      e_0 &\text{otherwise}
    \end{cases}
  \end{align*}
  From exercise 1 (general principle of recursive definition), the function $\rho$ above defines a unique function $h : J \to E$
  such that
  \begin{equation*}
    \forall \alpha \in J, \quad h ( \alpha ) = \rho ( h | S_\alpha )
    = \begin{cases}
      \min \big( E - h ( S_\alpha ) \big) &\text{if } h ( S_\alpha ) \neq E \\
      e_0 &\text{otherwise}
    \end{cases}
  \end{equation*}

  Let $J_0$ be the set of elements $\alpha$ of $J$ such that $h ( \alpha ) \leq k ( \alpha )$. Suppose that $S_\beta \subset J_0$
  for some $\beta \in J$. For all $\alpha \in S_\beta$, we have $h ( \alpha ) \leq k ( \alpha ) < k ( \beta )$, so $k ( \beta )$
  is an upper bound for $h ( S_\beta )$, and therefore an element of $E - h ( S_\beta )$. Since $h ( \beta )$ is the smallest
  element of $E - h ( S_\beta )$, we deduce that $h ( \beta ) \leq k ( \beta )$. Therefore $\beta \in J_0$, so $J_0$ is an inductive
  subset of $J$; thus $J_0 = J$, and
  \begin{equation*}
    \forall \alpha \in J, \quad h ( \alpha ) \leq k ( \alpha )
  \end{equation*}

  Suppose that there exists $\alpha \in J$ such that $h ( S_\alpha ) = E$. Since $k ( \alpha ) \in E$, we deduce that
  there exists $\beta \in S_\alpha$ such that $h ( \beta ) = k ( \alpha )$. From this, we get
  \begin{equation*}
    k ( \alpha ) = h ( \beta ) \leq k ( \beta ) < k ( \alpha )
  \end{equation*}
  This is a contradiction, so for all $\alpha \in J$, $h ( S_\alpha ) \neq E$.

  From the above we deduce that $h$ is defined by:
  \begin{equation*}
    \forall \alpha \in J, \quad h ( \alpha ) = \min \big( E - h ( S_\alpha ) \big)
  \end{equation*}
  We are therefore in the hypotheses of exercise 2, and conclude that $h$ is order-preserving and its image is $E$ or a section of $E$.
  As an order-preserving map, $h$ is injective, and therefore a bijection from $J$ to $h ( J )$. Thus $J$ has the order type of
  either $E$ or a section of $E$.

\end{proof}

\end{document}
